\documentclass[a4paper]{article} 
\addtolength{\hoffset}{-2.25cm}
\addtolength{\textwidth}{4.5cm}
\addtolength{\voffset}{-3.25cm}
\addtolength{\textheight}{5cm}
\setlength{\parskip}{0pt}
\setlength{\parindent}{0in}

%----------------------------------------------------------------------------------------
%	PACKAGES AND OTHER DOCUMENT CONFIGURATIONS
%----------------------------------------------------------------------------------------

\usepackage{blindtext} % Package to generate dummy text
% \usepackage{charter} % Use the Charter font
\usepackage[utf8]{inputenc} % Use UTF-8 encoding
\usepackage{microtype} % Slightly tweak font spacing for aesthetics
\usepackage[english, ngerman]{babel} % Language hyphenation and typographical rules
\usepackage{amsthm, amsmath, amssymb} % Mathematical typesetting
\usepackage{float} % Improved interface for floating objects
\usepackage[final, colorlinks = true, 
            linkcolor = black, 
            citecolor = black]{hyperref} % For hyperlinks in the PDF
\usepackage{graphicx, multicol} % Enhanced support for graphics
\usepackage{xcolor} % Driver-independent color extensions
\usepackage{marvosym, wasysym} % More symbols
\usepackage{rotating} % Rotation tools
\usepackage{censor} % Facilities for controlling restricted text
\usepackage{listings, style/lstlisting} % Environment for non-formatted code, !uses style file!
\usepackage{pseudocode} % Environment for specifying algorithms in a natural way
\usepackage{style/avm} % Environment for f-structures, !uses style file!
\usepackage{booktabs} % Enhances quality of tables
\usepackage{tikz-qtree} % Easy tree drawing tool
\tikzset{every tree node/.style={align=center,anchor=north},
         level distance=2cm} % Configuration for q-trees
\usepackage{style/btree} % Configuration for b-trees and b+-trees, !uses style file!
\usepackage[backend=biber,style=numeric,
            sorting=nyt]{biblatex} % Complete reimplementation of bibliographic facilities
\addbibresource{ecl.bib}
\usepackage{csquotes} % Context sensitive quotation facilities
\usepackage[ddmmyyyy]{datetime} % Uses YEAR-MONTH-DAY format for dates
%\renewcommand{\dateseparator}{-} % Sets dateseparator to '-'
\usepackage{fancyhdr} % Headers and footers
\pagestyle{fancy} % All pages have headers and footers
\fancyhead{}\renewcommand{\headrulewidth}{0pt} % Blank out the default header
\fancyfoot[L]{} % Custom footer text
\fancyfoot[C]{} % Custom footer text
\fancyfoot[R]{\thepage} % Custom footer text
\newcommand{\note}[1]{\marginpar{\scriptsize \textcolor{red}{#1}}} % Enables comments in red on margin

%----------------------------------------------------------------------------------------

\usepackage{mathtools}
\DeclarePairedDelimiter\ceil{\lceil}{\rceil}
\DeclarePairedDelimiter\floor{\lfloor}{\rfloor}
\begin{document}
	
	%-------------------------------
	%	TITLE SECTION
	%-------------------------------
	
	\fancyhead[C]{}
	\hrule \medskip % Upper rule
	\begin{minipage}{0.295\textwidth} 
		\raggedright
		\footnotesize
		Université de Lorraine\hfill\\   
		Télécom Nancy - 1A\hfill\\
		Omar CHIDA
	\end{minipage}
	\begin{minipage}{0.4\textwidth} 
		\centering 
		\large 
		Travaux Dirigés 1\\ 
		\normalsize 
		Langage C\\ 
	\end{minipage}
	\begin{minipage}{0.295\textwidth} 
		\raggedleft
		\today\hfill\\
	\end{minipage}
	\medskip\hrule 
	\bigskip
	
	%-------------------------------
	%	CONTENTS
	%-------------------------------
	
	\section*{Partie 1 : Les bases}
	- Le but de cette première section est de se familiariser avec les bases du C.\\
	- L'utilisation des fonctions déjà implémentées n'est pas autorisée !
	
	\subsection*{Mesure de la longueur des chaînes}
	\subsubsection*{Exercice 1 :}
	Dans cet exercice, nous allons essayer d'implémenter une fonction qui mesure la longueur d'une chaîne terminée par null. Cette fonction se comportera comme la fonction \texttt{strlen} de la bibliothèque standard C. On vous rappele que \texttt{strlen} prend un pointeur de char et renvoie un entier.
	Le profile de \texttt{strlen} :
	\begin{lstlisting}[language=C]
int strlen(char* str);
	\end{lstlisting}
	Implémentez une fonction appelée \texttt{len(str)} qui prend un pointeur de char comme argument et renvoie un entier positif. Vous ne devriez pas utiliser la fonction \texttt{strlen}.
	
	\subsection*{It's all binary}
	\subsubsection*{Exercice 2.1 : Leading zeros}
	Implémentez une fonction appelée \texttt{clz(bits)}\footnote{count leading zeros} qui prend un entier positif comme argument et compte le nombre de bits mis à zéro en allant de gauche à droite jusqu'à ce qu'il atteigne un 1 ou la fin. (Il compte le nombre de zéros avant la première occurrence d'un) \\
	Exemples:
\begin{lstlisting}[language=C]
Pour 00000000 00000000 00000000 00010000 clz doit retourner : 27
Pour 00000001 00000000 00000000 00010000 clz doit retourner : 7
Pour 00000000 00001000 11111111 00010000 clz doit retourner : 12
Pour 00000000 00000001 11111111 00010000 clz doit retourner : 15
\end{lstlisting}

	
	\subsubsection*{Exercice 2.2 : Nextpow2}
	Implémenter une fonction \texttt{nextpow2(n)} qui prend un entier positif et renvoie la puissance de 2 suivante la plus proche. \\
	Par exemple:
\begin{lstlisting}[language=C]
nextpow2(1) 	doit retourner 1
nextpow2(3) 	doit retourner 4
nextpow2(5) 	doit retourner 8
nextpow2(31) 	doit retourner 32
nextpow2(120) 	doit retourner 128
\end{lstlisting}

	\subsubsection*{Exercice 2.3 : Log2}
	Implémenter une fonction \texttt{log2(n)} qui prend un entier positif et renvoie un entier positive qui represente l'arrondi du log2 de l'entré.  La fonction peut être décrite par la formule suivante : $log2(x) = \ceil*{x}$. \\
	Par exemple:
	\begin{lstlisting}[language=C]
log2(1) 		doit retourner 0
log2(3) 		doit retourner 2
log2(5) 		doit retourner 3
log2(31) 		doit retourner 5
log2(120) 		doit retourner 7
	\end{lstlisting}

	\subsection*{PGCD}
	\subsubsection*{Exercice 3}
	Implémenter une fonction \texttt{gdc(a, b)} qui prend 2 entiers positifs et renvoie leur pgcd. \\
	Par exemple:
	\begin{lstlisting}[language=C]
gcd(1, 3) 		doit retourner 1
gcd(3, 5) 		doit retourner 1
gcd(6, 8) 		doit retourner 2
gcd(8, 32) 		doit retourner 8
gcd(30, 120) 	doit retourner 30
	\end{lstlisting}
	
	\section*{Partie 2 : }
	- Le but de cette partie est de traiter les boucles et les conditions et de se familiariser avec la fonction de \texttt{printf}.\\
	\subsection*{Pyramide en C}
	\subsubsection*{Exercice 4 : }
	Ecrivez un programme C qui affiche un pyramide de hauteur \texttt{n}. Pour ce faire, crée une fonction \texttt{pyramide} qui prend comme paramètre un entier \texttt{n} qui représente la hauteur de la pyramide et ne renvoie rien. L'argument n doit être passé à l'exécutable dans la ligne de commande.
	Exemples : \\
\begin{lstlisting}[language=C]
./pyramide 1
*
\end{lstlisting}
\begin{lstlisting}[language=C]
./pyramide 3
  *
 * *
* * *
\end{lstlisting}
\begin{lstlisting}[language=C]	
./pyramide 4
   *
  * *
 * * *
* * * *
\end{lstlisting}
	\subsection*{Triangle de Pascal}
	\subsubsection*{Exercice 5 : }
	Ecrivez un programme C, qui affiche un triangle de Pascale jusqu'au niveau \texttt{h}.
	
	\begin{lstlisting}[language=C]
./pascal 3
  1
 1 1
1 2 1
	\end{lstlisting}
	\begin{lstlisting}[language=C]	
./pyramide 4
   1
  1 1
 1 2 1
1 3 3 1
	\end{lstlisting}

	\section*{Partie 3 : }
	- Le but de cette partie est de manipuler les allocations dynamiques de mémoire.
	\subsection*{Tableau de Tableau (Les Matrices)}
	\subsubsection*{Exercice 6 : }
	Il existe 3 façons différentes de représenter une matrice en C. Créez trois fonctions appelées respectivement \texttt{mat1}, \texttt{mat2 } et \texttt{mat3} qui prend deux arguments N et M et créer une matrice de dimension NxM. \\
	N et M sont spécifiés à l'exécution via les arguments du programme.
\end{document}
