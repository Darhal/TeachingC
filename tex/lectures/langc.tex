  	\section{La langage C}
  	\subsection{Les bases}
  	\begin{frame}{In the beginning there was main}
  		\begin{block}{La fonction main}
  			La fonction \alert{main} est le point d'entrée du programme \footnote[frame]{l'exécutable}.
  		\end{block}
  		\begin{exampleblock}{Profils possibles :}
  			\begin{itemize}
  				\item \texttt{int main()}
  				\item \texttt{int main(int argc, char** argv)}
  			\end{itemize}
  		\end{exampleblock}
  		\begin{alertblock}{Profils qui compilent mais avec un Warning:}
  			\begin{itemize}
	  			\item \texttt{void main()}
	  			\item \texttt{void main(int argc, char** argv)}
  			\end{itemize}
  		\end{alertblock}
  	\end{frame}
  
  	\begin{frame}{In the beginning there was main}
		\framesubtitle{Les arguments de main}
		\begin{itemize}
			\item \alert{argc} : Indique le nombre d'arguments passés au programme. La valeur minimale de \texttt{argc} est 1 car le premier argument est toujours le nom du programme.
			\item \alert{argv} : Un tableau de chaîne contenant les arguments passés au programme, \texttt{argv[0]} est le nom du programme, \texttt{argv[1]} est le nom du premier argument, et ainsi de suite.
		\end{itemize}
  	\end{frame}
  
  	\begin{frame}{In the beginning there was main}
	  	\begin{exampleblock}{Exemple :}
	  		Soit la commande suivante :~~"\texttt{./a.out abc w 23 1}" \\
	  		- \texttt{argc} : vaut 5 \\
	  		- \texttt{argv[0]} : est la chaine "./a.out" \\
	  		- \texttt{argv[1]} : est la chaine "abc" \\
	  		- \texttt{argv[2]} : est la chaine "w" \\
	  		- \texttt{argv[3]} : est la chaine "23" \\
	  		- \texttt{argv[4]} : est la chaine "1" \\
	  	\end{exampleblock}
  	\end{frame}
  
  	\begin{frame}{Les types de base}
  		\begin{table}[!b]
  			{\carlitoTLF % Use monospaced lining figures
  			\begin{tabularx}{\textwidth}{Xrrr}
  				\textbf{Type} & \textbf{Taille min} & \textbf{Intervalle} & \textbf{Spécificateur de format} \\
  				\toprule
  				\texttt{char}      & 1o  & -127..127  		   & \texttt{\%c}    				    \\
  				\texttt{short}     & 2o  & -32767..32767  	   & \texttt{\%c} ou \texttt{\%hhi}     \\
  				\texttt{int}       & 4o  & $-2^{31}$..$2^{31}$ & \texttt{\%d}    				    \\
  				\texttt{long long} & 8o  & $-2^{63}$..$2^{63}$ & \texttt{\%lld}  				    \\
  				\texttt{float}     & 4o  &    ..    		   & \texttt{\%f}    				    \\
  				\texttt{double}    & 8o  &    ..   			   & \texttt{\%lf}   					\\
  				\bottomrule
  			\end{tabularx}}
  			\caption{Les types de base signés en C}
  		\end{table}
  	\end{frame}
  
  	\begin{frame}{Les types de base}
  		\begin{table}[!b]
  			{\carlitoTLF % Use monospaced lining figures
  				\begin{tabularx}{\textwidth}{Xrrr}
  					\textbf{Type} & \textbf{Taille min} & \textbf{Intervalle} & \textbf{Spécificateur de format} \\
  					\toprule
  					\texttt{unsigned char}      & 1o  & 0..255  		    & \texttt{\%c}    				    \\
  					\texttt{unsigned short}     & 2o  & 0..65535  	   		& \texttt{\%c} ou \texttt{\%hhu}    \\
  					\texttt{unsigned int}       & 4o  & 0..$2^{32}-1$ 		& \texttt{\%u}    				    \\
  					\texttt{unsigned long long} & 8o  & 0..$2^{64}-1$ 		& \texttt{\%llu}  				    \\
  					\bottomrule
  			\end{tabularx}}
  			\caption{Les types de base non-signés en C}
  		\end{table}
  	\end{frame}
 
\defverbatim[colored]\ifsignle{
\begin{lstlisting}[language=C,tabsize=2]
if (some_condition)
	statment; // Une seule instruction, cad un seul point-virgule	
\end{lstlisting}}
  
\defverbatim[colored]\ifmulti{
\begin{lstlisting}[language=C,tabsize=2]
if (some_condition) {
	statment_1;
	statment_2;
	// ...
	statment_N;
}
\end{lstlisting}}

\defverbatim[colored]\ifelsesignle{
\begin{lstlisting}[language=C,tabsize=2]
if (some_condition)
	statment; // Une seule instruction, cad un seul point-virgule	
[[else
	statment2; // Un seul point-virgule	
]]
\end{lstlisting}}

\defverbatim[colored]\ifelsemulti{
\begin{lstlisting}[language=C,tabsize=2]
if (some_condition1) {
	statment_1;
	// ...
	statment_N;
} [[ else if (some_condition2) {
	statment_1;
	// ...
	statment_N;
// Possibilite d'ajouter plusieurs blocs else if 
} ]] [[ else {
	statment_1;
	// ...
	statment_N;
} ]]
\end{lstlisting}}
  	\begin{frame}{Les conditions}
  		\begin{block}{Syntaxe : Première possibilité}
  			\ifelsesignle
  		\end{block}
		\begin{alertblock}{N.B. :}
			Ce qui est mis entre $\big[\big[~~...~~\big]\big]$ est facultatif.
		\end{alertblock}
  	\end{frame}
  	
	\begin{frame}{Les conditions}
		  		\begin{block}{Syntaxe : Deuxième possibilité}
			\ifelsemulti
		\end{block}
	\end{frame}

\defverbatim[colored]\ifexampleone{
\begin{lstlisting}[language=C,tabsize=2]
int i = 0;
if (i--)
	puts("Hello World");
\end{lstlisting}}

\defverbatim[colored]\ifexampletwo{
\begin{lstlisting}[language=C,tabsize=2]
int i = -1;
if (i++)
	puts("Hello World");
\end{lstlisting}}

\defverbatim[colored]\ifexamplethree{
\begin{lstlisting}[language=C,tabsize=2]
int i = -1;
if (i++)
	if (++i)
		if ('c')
			puts("Hello World");
\end{lstlisting}}

	\begin{frame}{Les conditions}
		\begin{block}{Comment une condition est évaluée  ?}
			Le type \alert{booléen } n'existe pas en C. Si une expression est évaluée à 0, elle est considérée comme \alert{False}, sinon elle est considérée comme \alert{True}.
		\end{block}
	\end{frame}

	\begin{frame}{Les conditions : Exemples}	
		\begin{center}
			\begin{minipage}[t]{0.48\linewidth}
				\text{Exemple 1 :}
				\ifexampleone
			\end{minipage}
			\qquad
			\begin{minipage}[t]{0.48\linewidth}
				\text{Exemple 2 :}
				\ifexampletwo
			\end{minipage}
			\begin{minipage}[t]{0.48\linewidth}
				\text{Exemple 3 :}
				\ifexamplethree
			\end{minipage}
		\end{center}
	\end{frame}

	\begin{frame}{Les conditions : Exemples}	
		\text{Exemple 1 : \alert{(N'affiche rien)}}
		\ifexampleone
		\text{Exemple 2 : \alert{(Affiche "Hello World")}}
		\ifexampletwo
		\text{Exemple 3 : \alert{(Affiche "Hello World")}}
		\ifexamplethree
	\end{frame}
	
\defverbatim[colored]\switchSyntax{
\begin{lstlisting}[language=C,tabsize=4]
switch(expression) {
	case constant_expression1:
		statment_1;
		// optionally other statments ...
		break;    // optional
	case constant_expression2:
		statment_2;
		// optionally other statments ...
		break; 	  // optional
	// ...
	default: 	  // optional
		statment_3;
		// optionally other statments ...
}
\end{lstlisting}}
	\begin{frame}{Les conditions : switch}
		\switchSyntax
	\end{frame}

\defverbatim[colored]\forsyntax{
\begin{lstlisting}[language=C,tabsize=2]
for (initialisation; condition; increment) {
	// ...
}
\end{lstlisting}}
\defverbatim[colored]\forsyntaxtwo{
\begin{lstlisting}[language=C,tabsize=2]
for (initialisation; condition; increment)
	statment;
\end{lstlisting}}

\defverbatim[colored]\forWrittenUsingWhile{
\begin{lstlisting}[language=C,tabsize=2]
initialisation;
while (condition) {
	// ...
	increment;
}
\end{lstlisting}}

	\begin{frame}{Les boucles}	
		\begin{block}{Syntaxe : boucle pour}
			\forsyntax
			L'instruction \alert{d'initialisation } n'est exécutée qu'au début de la boucle. La \alert{condition} est vérifiée à chaque itération, \alert{l'instruction d'incrémentation} est également exécutée à chaque itération.
		\end{block}
		\begin{alertblock}{Une boucle for peut être écrite comme une boucle while}
			\forWrittenUsingWhile
		\end{alertblock}
	\end{frame}

\defverbatim[colored]\forExmpOne{
\begin{lstlisting}[language=C,tabsize=2]
for (int i = 0; i < 10; i++)
	for (int j = 0; j < 20; j++)
		puts("Hello World");
\end{lstlisting}}

\defverbatim[colored]\forExmpTwo{
\begin{lstlisting}[language=C,tabsize=2]
for (;;)
	puts("Hello World");
\end{lstlisting}}

	\begin{frame}{Les boucles}
		\begin{block}{Syntaxe : boucle pour}
			Comme la syntaxe du \texttt{if}, la boucle pour peut être écrite de cette manière:
			\forsyntaxtwo
		\end{block}
		\begin{center}
			\begin{minipage}[t]{0.48\linewidth}
				\text{Example 1:}
				\forExmpOne
			\end{minipage}
			\begin{minipage}[t]{0.48\linewidth}
				\text{Example 2:}
				\forExmpTwo
			\end{minipage}
		\end{center}
	\end{frame}
	

\defverbatim[colored]\forExmpThree{
\begin{lstlisting}[language=C,tabsize=2]
for (int i = -1; i < 10; i++) {
	break;
	printf("Hello World\n");
}
\end{lstlisting}}

\defverbatim[colored]\forExmpFour{
\begin{lstlisting}[language=C,tabsize=2]
for (int i = -1; i < 10; i++) {
	if (i > 3) continue;
	printf("Hello World\n");
}
\end{lstlisting}}

\defverbatim[colored]\forExmpFive{
\begin{lstlisting}[language=C,tabsize=2]
for (int i = -1; i < 10; i++) {
	continue;
	printf("Hello World\n");
}
\end{lstlisting}}
	
	\begin{frame}{Les boucles}
		\begin{center}
			\begin{minipage}[t]{0.8\linewidth}
				Exemple 1: (\alert{Affiche 200 "Hello World"})
				\forExmpOne
			\end{minipage}
			\begin{minipage}[t]{0.8\linewidth}
				Exemple 2: (\alert{Affiche une infinité de "Hello World"})
				\forExmpTwo
			\end{minipage}
			\begin{minipage}[t]{0.8\linewidth}
				Exemple 3:
				\forExmpThree
			\end{minipage}
		\end{center}
	\end{frame}
	
	\begin{frame}{Les boucles}	
		\begin{minipage}[t]{0.8\linewidth}
			Exemple 3: (\alert{N'affiche rien})
			\forExmpThree
		\end{minipage}
		\begin{minipage}[t]{0.8\linewidth}
			Exemple 4:
			\forExmpFour
		\end{minipage}
		\begin{minipage}[t]{0.8\linewidth}
			Exemple 5:
			\forExmpFive
		\end{minipage}
	\end{frame}

	\begin{frame}{Les boucles}
		\begin{minipage}[t]{0.8\linewidth}
			Exemple 4 :  (\alert{Affiche 5 "Hello World"})
			\forExmpFour
		\end{minipage}
		\begin{minipage}[t]{0.8\linewidth}
			Exemple 5: (\alert{N'affiche rien})
			\forExmpFive
		\end{minipage}
	\end{frame}

\defverbatim[colored]\WhileSyntax{
\begin{lstlisting}[language=C,tabsize=2]
while (condition) {
	// ..
};
\end{lstlisting}}

\defverbatim[colored]\WhileInfinite{
\begin{lstlisting}[language=C,tabsize=2]
while (1) {
	// ..
};
\end{lstlisting}}

	\begin{frame}{Les boucles}
		\begin{block}{Syntaxe : boucle tantque}
			\WhileSyntax
			La boucle continue de s'exécuter jusqu'à ce que la condition soit \alert{fausse}.
		\end{block}
		\begin{exampleblock}{Example d'une boucle infinie :}
			\WhileInfinite
		\end{exampleblock}
	\end{frame}

\defverbatim[colored]\doWhileSyntax{
\begin{lstlisting}[language=C,tabsize=2]
do {
	// ..
} while(condition);
\end{lstlisting}}
	\begin{frame}{Les boucles}
		\begin{block}{Syntaxe : boucle faire ... tantque}
			\doWhileSyntax
			La boucle continue de s'exécuter jusqu'à ce que la condition soit \alert{fausse}. 
			Cette condition est similaire à une boucle tantque, malgrée le fait qu'elle est garantie de s'exécuter au moins une fois.
		\end{block}
	\end{frame}
	
\defverbatim[colored]\structSyntax{
\begin{lstlisting}[language=C,tabsize=2]
struct StructName 
{
	TypenameA field1_name;
	TypenameB field2_name;
	TypenameC field3_name;
	// ...
};
\end{lstlisting}}

\defverbatim[colored]\structExmp{
\begin{lstlisting}[language=C,tabsize=2]
struct A 
{
	int a; // sizeof(int) = 4
	short b; // sizeof(short) = 2
	double b; // sizeof(double) = 8
	char str[256]; // sizeof(char) * 256 = 1 * 256 elements
};	
\end{lstlisting}}

  	\subsection{Types définis par l'utilisateur: struct, union, enum}
  	\begin{frame}{Les structs}
  		\begin{block}{Définition et Syntaxe :}
  			Struct, une abréviation de structure, est un type défini par l'utilisateur qui est composé d'autres types qui peuvent ou non être fondamentaux.
  			\structSyntax
  		\end{block}
  	\end{frame}
  
  	\begin{frame}{Les structs}
  		\begin{alertblock}{Quelques remarques :}
  			- La taille d'une structure est la somme de la taille de ses champs. \\
  			- La taille est accessible en utilisant \alert{\texttt{sizeof(struct StructName)}}.
  		\end{alertblock}
  		\begin{exampleblock}{Exeemple:}
  			\structExmp
  			La taille est: \texttt{sizeof(struct A)} = $4 + 2 + 8 + 256 = 270$ octets.
  		\end{exampleblock}
  	\end{frame}
  
\defverbatim[colored]\unionSyntax{
\begin{lstlisting}[language=C,tabsize=2]
union UnionName 
{
	TypenameA field1_name;
	TypenameB field2_name;
	TypenameC field3_name;
	// ...
};
\end{lstlisting}}

\defverbatim[colored]\unionExmp{
\begin{lstlisting}[language=C,tabsize=2]
union A 
{
	int a; // sizeof(int) = 4
	short b; // sizeof(short) = 2
	double b; // sizeof(double) = 8
	char str[256]; // sizeof(char) * 256 = 1 * 256 elements
};	
\end{lstlisting}}

\defverbatim[colored]\unionExmpDanger{
\begin{lstlisting}[language=C,tabsize=2]
union B 
{
	int a;
	short b;
	char str[4];
};
union B var;
var.str[0] = 'T';
var.str[1] = 'N';
var.str[2] = 'C';	
var.str[3] = 'Y';
var.b = 256; // ATTENTION: var.str ne vaut plus TNCY !!!
\end{lstlisting}}
 
  	\begin{frame}{Les unions}
  		\begin{block}{Définition et Syntaxe :}
  			L'union est un type défini par l'utilisateur qui est composé d'autres types qui peuvent ou non être fondamentaux. La mémoire réelle allouée à une union est égale au maximum de ses champs. Tous les champs d'un union partagent donc la même mémoire sous-jacente.
  			\unionSyntax
  		\end{block}

  	\end{frame}
  	
  	\begin{frame}{Les unions}
  		\begin{alertblock}{Quelques remarques:}
  			- La taille d'une union est le maximum des tailles de ses champs. \\
  			- La taille est accessible en utilisant \alert{\texttt{sizeof(union UnionName)}}. \\
  		\end{alertblock}
  		\begin{exampleblock}{Exemple:}
  			\unionExmp
  			La taille est: \texttt{sizeof(union A)} = $max(4, 2, 8, 256) = 256$ octets.
  		\end{exampleblock}
  	\end{frame}

	\begin{frame}{Les unions}
		\begin{alertblock}{ATTENTION : Soyez prudent lorsque vous accédez aux champs d'union. Écrire dans n'importe quel champ d'union peut écraser la mémoire déjà écrite par un autre champ.}
		\unionExmpDanger
		\end{alertblock}
	\end{frame}
	
	    
	\defverbatim[colored]\enumSyntax{	
	\begin{lstlisting}[language=C,tabsize=4]
enum EnumName {
	OptionName1,
	OptionName2,
	OptionName3,
	...
};
	\end{lstlisting}}

	\begin{frame}{Les enums}  	
		\begin{block}{Définition :}
			L'énumération (ou enum) est un type de données défini par l'utilisateur. Il est principalement utilisé pour attribuer des noms à des \alert{constantes intégrales}\footnote[frame]{Des constantes de type intégrale (fondamental)}. Ceci est destiné à augmenter la lisibilité et la maintenabilité du programme
		\end{block}
		\begin{block}{Syntaxe :}
			\enumSyntax
		\end{block}
	\end{frame}
	
	\defverbatim[colored]\enumOne{	
	\begin{lstlisting}[language=C,tabsize=4]
enum MyEnum { A, B, C };
	\end{lstlisting}}  

	\begin{frame}{Les enums}
		\begin{alertblock}{Quelques remarques :}
			\begin{itemize}
				\item Pour avoir  un bon style de programmation, les constantes d'énumération doivent être écrites en majuscules comme toutes les autres constantes de programme.
				\item Puisque les énumerations sont des types défini par l'utilisateur, des variables peuvent être déclarées en utilisant ce type.
			\end{itemize}
			Si une énumération est définie comme suit: \enumOne alors \texttt{A} sera 0, \texttt{B} sera 1 et ainsi de suite.
		\end{alertblock}
	\end{frame}
	
	\defverbatim[colored]\enumTwo{	
		\begin{lstlisting}[language=C,tabsize=4]
enum TrafficLight { 
	RED = 0, // = 0
	ORANGE,  // = ??
	GREEN    // = ??
};
	\end{lstlisting}}
	\defverbatim[colored]\enumThree{	
	\begin{lstlisting}[language=C,tabsize=4]
enum TrafficLight { 
	RED , 		// = ??
	ORANGE = 1, // = 1
	GREEN    	// = ??
};
	\end{lstlisting}}
	\defverbatim[colored]\enumFour{	
		\begin{lstlisting}[language=C,tabsize=4]
enum TrafficLight { 
	RED, 		// = ??
	ORANGE, 	// = ??
	GREEN = 5,  // = 5
	BLUE,		// = ??
};
	\end{lstlisting}}
	\begin{frame}{Les enums}
		Example 1 : 
		\enumTwo
		Example 2 : 
		\enumThree
		
	\end{frame}
	
	\defverbatim[colored]\enumTwoSolution{	
	\begin{lstlisting}[language=C,tabsize=4]
enum TrafficLight { 
	RED = 0, // = 0
	ORANGE,  // = 1
	GREEN    // = 2
};
	\end{lstlisting}}
	\defverbatim[colored]\enumThreeSolution{	
	\begin{lstlisting}[language=C,tabsize=4]
enum TrafficLight { 
	RED, 		// = 0
	ORANGE = 1, // = 1
	GREEN    	// = 2
};
	\end{lstlisting}}
	\defverbatim[colored]\enumFourSolution{	
	\begin{lstlisting}[language=C,tabsize=4]
enum TrafficLight { 
	RED, 		// = 0
	ORANGE, 	// = 0
	GREEN = 5,  // = 5
	BLUE     	// = 6
};
	\end{lstlisting}}
	
	\defverbatim[colored]\enumFive{	
	\begin{lstlisting}[language=C,tabsize=4]
enum TrafficLight myVar = BLUE; // = ??
myVar = ORANGE; // = ??
myVar = GREEN + 1; // = ??
myVar = GREEN + BLUE + ORANGE; // = ??
printf("%d\n", myVar);
printf("%d\n", RED);
	\end{lstlisting}}
	\defverbatim[colored]\enumFiveSolution{	
	\begin{lstlisting}[language=C,tabsize=4]
enum TrafficLight myVar = BLUE; // = 6
myVar = ORANGE; // = 0
myVar = GREEN + 1; // = 6
myVar = GREEN + BLUE + ORANGE; // = 11
printf("%d\n", myVar); // prints 11
printf("%d\n", RED); // prints 0
	\end{lstlisting}}
	\begin{frame}{Les enums}
		Solution 1 : 
		\enumTwoSolution
		Solution 2 : 
		\enumThreeSolution
	\end{frame}
	
	\begin{frame}{Les enums}
		Example 3 : 
		\enumFour
		Example 4 : 
		\enumFive
	\end{frame}
	
	\begin{frame}{Les enums}
		Solution 3 : 
		\enumFourSolution
		Solution 4 : 
		\enumFiveSolution
	\end{frame}


  	%%%%%% Les Tableaux %%%%%%
\defverbatim[colored]\arrayDecl{
\begin{lstlisting}[language=C,tabsize=2]
Typename ArrayName[Capacity]; // Where capacity is a positive 
                              // integer
\end{lstlisting}}

\defverbatim[colored]\arrayDeclExmpl{
\begin{lstlisting}[language=C,tabsize=2]
int tab[8];	// tab contains 8 elments of type int
\end{lstlisting}}

  	\subsection{Les tableaux}
  	\begin{frame}{Les tableaux}
  		\begin{block}{Définition et Syntaxe}
  			Un tableau est une collection d'éléments du même type qui sont stockés en mémoire de manière contegieuse. Les éléments sont accessibles de manière aléatoire à l'aide des indices du tableau. \\
  			Un tableau peut être \alert{déclaré} de cette manière en C:
  			\arrayDecl
  		\end{block}
  		\begin{block}{Exemple :}
  			\texttt{tab} contient 8 éléments de type \texttt{int} indexables de $0$ à $7$ : 
  			\arrayDeclExmpl
  		\end{block}
  	\end{frame}
  
  	\begin{frame}{Les tableaux}
  		\framesubtitle{Représentation des tableaux en mémoire}
  		\begin{figure}[!h]
  			\centering
	    	\begin{tikzpicture} [nodes in empty cells,
				nodes={minimum width=1cm, minimum height=1cm},
				row sep=-\pgflinewidth, column sep=-\pgflinewidth]
				% border/.style={draw}
				\matrix(vector)[matrix of nodes, ampersand replacement=\&, % <- added ampersand replacement
				row 1/.style={nodes={draw=none, minimum width=1cm, fill=fibeamer@black}},
				nodes={freestruct, anchor=center}]
				{ % use \& instead of & as column separator
					{0} \& {1} \& {2} \& {3} \& {...} \& {N}\\
					$e_{0}$ \& $e_{1}$ \& $e_{2}$ \& $e_{3}$ \& $...$ \& $e_{N}$\\
				};
				
				\draw (-3,-2.3) node{Début du tableau};
				\draw[-{Latex[length=2.5mm]}] (-3,-2) -- (-3,-1);
				
				\draw (2.5,-2.3) node{Dernier élément};
				\draw[-{Latex[length=2.5mm]}] (2.5,-2) -- (2.5,-1);
				
				\draw (0,-3.8) node{Figure - Représentation mémoire de tab};
			\end{tikzpicture}
		\end{figure}
	\end{frame}

	\begin{frame}{Les tableaux}
		\framesubtitle{Exemple: Représentation des tableaux en mémoire}
		\begin{block}{Exemple :}
			\texttt{tab} contient 8 éléments de type \texttt{int} indexables de $0$ à $7$ : 
			\arrayDeclExmpl
		\end{block}
		\begin{figure}[!h]
			\centering
			\begin{tikzpicture} [nodes in empty cells,
				nodes={minimum width=1cm, minimum height=1cm},
				row sep=-\pgflinewidth, column sep=-\pgflinewidth]
				% border/.style={draw}
				\matrix(vector)[matrix of nodes, ampersand replacement=\&, % <- added ampersand replacement
				row 1/.style={nodes={draw=none, minimum width=1cm, fill=fibeamer@black}},
				nodes={freestruct, anchor=center}]
				{ % use \& instead of & as column separator
					{0} \& {1} \& {2} \& {3} \& {4} \& {5} \& {6} \& {7}\\
					$2$ \& $-52$ \& $256$ \& $42$ \& $87$ \& $356$ \& $80$ \& $-356$\\
				};
				
				\draw (-4.06,-2.3) node{tab};
				\draw[-{Latex[length=3mm]}] (-4.05,-2) -- (-4.06,-1);
			\end{tikzpicture}
		\end{figure}
	\end{frame}


\defverbatim[colored]\arrayInitExmplOnePrime{
\begin{lstlisting}[language=C,tabsize=2]
int tab[8] = { 0 };
\end{lstlisting}}

\defverbatim[colored]\arrayInitExmplOne{
\begin{lstlisting}[language=C,tabsize=2]
int tab[8] = {};
\end{lstlisting}}

	\begin{frame}{Les tableaux}
		\framesubtitle{Initialisation des tableaux}
		\begin{block}{Exemple :}
			Le code suivant initialisera à zéro tous les éléments du tableau\footnote[frame]{Ce type d'initialisation s'appelle \alert{Zero-Initialization} et peut également être utilisé avec les structures et les unions} :
			\arrayInitExmplOne
		\end{block}
		\begin{figure}[!h]
			\centering
			\begin{tikzpicture} [nodes in empty cells,
				nodes={minimum width=0.7cm, minimum height=0.7cm},
				row sep=-\pgflinewidth, column sep=-\pgflinewidth]
				% border/.style={draw}
				\matrix(vector)[matrix of nodes, ampersand replacement=\&, % <- added ampersand replacement
				row 1/.style={nodes={draw=none, minimum width=0.7cm, fill=fibeamer@black}},
				nodes={freestruct, anchor=center}]
				{ % use \& instead of & as column separator
					{0} \& {1} \& {2} \& {3} \& {4} \& {5} \& {6} \& {7}\\
					$0$ \& $0$ \& $0$ \& $0$ \& $0$ \& $0$ \& $0$ \& $0$\\
				};
				
				\draw (-2.8,-1.7) node{tab};
				\draw[-{Latex[length=2.25mm]}] (-2.8,-1.5) -- (-2.8,-0.7);
				
				\draw (-0,-2.75) node{Figure - Représentation mémoire de tab};
			\end{tikzpicture}
		\end{figure}
	\end{frame}

\defverbatim[colored]\arrayInitExmplTwo{
\begin{lstlisting}[language=C,tabsize=2]
int tab[8] = { 1 };
\end{lstlisting}}

	\begin{frame}{Les tableaux}
		\framesubtitle{Initialisation des tableaux}
		\begin{block}{Exemple :}
			Le ci-dessous intilise le premier élément à 1 et le reste à 0 :
			\arrayInitExmplTwo
		\end{block}
		\begin{figure}[!h]
			\centering
			\begin{tikzpicture} [nodes in empty cells,
				nodes={minimum width=0.7cm, minimum height=0.7cm},
				row sep=-\pgflinewidth, column sep=-\pgflinewidth]
				% border/.style={draw}
				\matrix(vector)[matrix of nodes, ampersand replacement=\&, % <- added ampersand replacement
				row 1/.style={nodes={draw=none, minimum width=0.7cm, fill=fibeamer@black}},
				nodes={freestruct, anchor=center}]
				{ % use \& instead of & as column separator
					{0} \& {1} \& {2} \& {3} \& {4} \& {5} \& {6} \& {7}\\
					$1$ \& $0$ \& $0$ \& $0$ \& $0$ \& $0$ \& $0$ \& $0$\\
				};
				
				\draw (-2.8,-1.7) node{tab};
				\draw[-{Latex[length=2.25mm]}] (-2.8,-1.5) -- (-2.8,-0.7);
				
				\draw (-0,-2.75) node{Figure - Représentation mémoire de tab};
			\end{tikzpicture}
		\end{figure}
	\end{frame}

\defverbatim[colored]\arrayInitExmplThree{
\begin{lstlisting}[language=C,tabsize=2]
int tab[8] = { 1, 2, 3 };
\end{lstlisting}}

	\begin{frame}{Les tableaux}
		\framesubtitle{Initialisation des tableaux}
		\begin{block}{Exemple :}
			Le code ci-dessous initialise les 3 premiers éléments à 1, 2 et 3 respectivement et le reste à 0 :
			\arrayInitExmplThree
		\end{block}
		\begin{figure}[!h]
			\centering
			\begin{tikzpicture} [nodes in empty cells,
				nodes={minimum width=0.7cm, minimum height=0.7cm},
				row sep=-\pgflinewidth, column sep=-\pgflinewidth]
				% border/.style={draw}
				\matrix(vector)[matrix of nodes, ampersand replacement=\&, % <- added ampersand replacement
				row 1/.style={nodes={draw=none, minimum width=0.7cm, fill=fibeamer@black}},
				nodes={freestruct, anchor=center}]
				{ % use \& instead of & as column separator
					{0} \& {1} \& {2} \& {3} \& {4} \& {5} \& {6} \& {7}\\
					$1$ \& $2$ \& $3$ \& $0$ \& $0$ \& $0$ \& $0$ \& $0$\\
				};
				
				\draw (-2.8,-1.7) node{tab};
				\draw[-{Latex[length=2.25mm]}] (-2.8,-1.5) -- (-2.8,-0.7);
				
				\draw (-0,-2.75) node{Figure - Représentation mémoire de tab};
			\end{tikzpicture}
		\end{figure}
	\end{frame}
	
\defverbatim[colored]\arrayInitExmplFour{
\begin{lstlisting}[language=C,tabsize=2]
int tab[] = { 1, 2, 3, 4 }; // The size is 4, implicitly 
                            // calculated during compilation
\end{lstlisting}}

	\begin{frame}{Les tableaux}
		\begin{block}{Déduction de la taille du tableau}
			Une autre façon de déclarer et d'initialiser les tableaux en même temps :
			\arrayInitExmplFour
			Le compilateur calculera implicitement la taille du tableau lors de la compilation.
		\end{block}
		\begin{figure}[!h]
			\centering
			\begin{tikzpicture} [nodes in empty cells,
				nodes={minimum width=0.7cm, minimum height=0.7cm},
				row sep=-\pgflinewidth, column sep=-\pgflinewidth]
				% border/.style={draw}
				\matrix(vector)[matrix of nodes, ampersand replacement=\&, % <- added ampersand replacement
				row 1/.style={nodes={draw=none, minimum width=0.7cm, fill=fibeamer@black}},
				nodes={freestruct, anchor=center}]
				{ % use \& instead of & as column separator
					{0} \& {1} \& {2} \& {3} \\
					$1$ \& $2$ \& $3$ \& $4$ \\
				};
				
				\draw (-3,-0.35) node{tab};
				\draw[-{Latex[length=2.4mm]}, thick] (-3,-0.55) .. controls (-3,-1.35) and (-2,-1.35) .. (-1.4,-0.7);
				
				\draw (-0,-2.2) node{Figure - Représentation mémoire de tab};
			\end{tikzpicture}
		\end{figure}
	\end{frame}

\defverbatim[colored]\arraySizeOf{
\begin{lstlisting}[language=C,tabsize=2]
int tab[] = { 1, 2, 3, 4 };  
assert(sizeof(tab) == 16); // true
assert(sizeof(tab) == 4 * sizeof(int)); // true
assert(sizeof(tab)/sizeof(tab[0]) == 4); // true
\end{lstlisting}}
	\begin{frame}{Les tableaux}
		\begin{block}{Utilisation de l'opérateur \texttt{sizeof}}
			L'opérateur \alert{sizeof} peut être utilisé sur les tables déclarées \alert{statiquement} pour déterminer leur taille totale en \alert{octets}.
		\end{block}
		\begin{exampleblock}{Example :}
			\arraySizeOf
		\end{exampleblock}
	\end{frame}

\defverbatim[colored]\arraySizeOfExmplOne{
\begin{lstlisting}[language=C,tabsize=2]
struct Point
{
	int x, y;
};

struct Point triangle[] = { {0, 0}, {3, -4}, {5, -6} };  
// sizeof(struct Point) = ??
// sizeof(triangle) = ??
// sizeof(triangle) / sizeof(struct Point) = ??
\end{lstlisting}}
\defverbatim[colored]\arraySizeOfExmplOneSolution{
\begin{lstlisting}[language=C,tabsize=2]
struct Point
{
	int x, y;
};
		
struct Point triangle[] = { {0, 0}, {3, -4}, {5, -6} };  
// sizeof(struct Point) = 2 * sizeof(int) = 8
// sizeof(triangle) = 3 * sizeof(struct Point) = 24
// sizeof(triangle) / sizeof(struct Point) = 24 / 8 = 3
\end{lstlisting}}

  	\begin{frame}{Les tableaux}
  		Example 1 :
  		\arraySizeOfExmplOne
  	\end{frame}
  
  	\begin{frame}{Les tableaux}
  		Solution 1 :
  		\arraySizeOfExmplOneSolution
  	\end{frame}
  
\defverbatim[colored]\arraySizeOfExmplTwo{
\begin{lstlisting}[language=C,tabsize=2]
union ieee754 
{
	double unused;
	float f;
	unsigned int d;
};

union ieee754 integer_rep[] = { {1}, {.f=3.14}, {.d=42} };
// sizeof(union ieee754) = ??
// sizeof(integer_rep) = ??
// sizeof(integer_rep) / sizeof(union ieee754) = ??
\end{lstlisting}}

\defverbatim[colored]\arraySizeOfExmplTwoSolution{
\begin{lstlisting}[language=C,tabsize=2]
union ieee754 
{
	double unused;
	float f;
	unsigned int d;
};
		
union ieee754 integer_rep[] = { { 1 }, { .f=3.14 }, { .d=42 } };  
// sizeof(union ieee754) = max(8, 4, 4) = 8
// sizeof(integer_rep) = 8 * 3 = 24
// sizeof(integer_rep) / sizeof(union ieee754) = 24 / 8 = 3
\end{lstlisting}}
  	\begin{frame}{Les tableaux}
  		Example 2 :
  		\arraySizeOfExmplTwo
  	\end{frame}
    \begin{frame}{Les tableaux}
	  	Solution 2 :
	  	\arraySizeOfExmplTwoSolution
  	\end{frame}

\defverbatim[colored]\arraySizeOfExmplThree{
\begin{lstlisting}[language=C,tabsize=2]
void foo(int tab[]) {
	// sizeof(tab) = ??
}
void bar(int* tab) {
	// sizeof(tab) = ??
}
int main() {
	int tab[] = {1, 4, 8};
	int tab2[32];
	// sizeof(tab) = ?? 
	// sizeof(tab2) = ??
	int* tabcopy = tab;
	// sizeof(tabcopy) = ??
	foo(tab);
	bar(tab2);
}
\end{lstlisting}}

\defverbatim[colored]\arraySizeOfExmplThreeSolution{
\begin{lstlisting}[language=C,tabsize=2]
void foo(int tab[]) {
	// sizeof(tab) = 4 or 8 depending on the architecture
}
void bar(int* tab) {
	// sizeof(tab) = 4 or 8 depending on the architecture
}
int main() {
	int tab[] = {1, 4, 8};
	int tab2[32];
	// sizeof(tab) = 3 * 4 = 12
	// sizeof(tab2) = 32 * 4 = 128
	int* tabcopy = tab;
	// sizeof(tabcopy) = 4 or 8 depending on the architecture
	foo(tab);
	bar(tab2);
}
\end{lstlisting}}
  	\begin{frame}{Les tableaux}
  		\framesubtitle{Example 3 : }
  		\arraySizeOfExmplThree
  	\end{frame}
  	\begin{frame}{Les tableaux}
  		\framesubtitle{Solution 3 : }
  		\arraySizeOfExmplThreeSolution
  	\end{frame}
  
  	\begin{frame}{Les tableaux}
  		\begin{alertblock}{Attention au désintégration (decay) !}
  			Une variable de type tableau se \alert{désintègre}\footnote[frame]{Ce problème s'appelle "Array To Pointer Decay"} en pointeur lorsqu'elle est passée à une fonction en tant qu'argument ou copiée dans une autre variable. \\ Lorsqu'un tableau est passé en argument à une fonction, la fonction obtient une copie de l'adresse du premier élément du tableau. \\ 
  			$\implies$ Il y a une perte d'informations sur la taille du tableau d'ou le terme \alert{decay}.
  		\end{alertblock}
  	\end{frame}
  
\defverbatim[colored]\arrayIndexAccess{
\begin{lstlisting}[language=C,tabsize=2]
tab[i];       // reading from the ith element
tab[i] = ...; // writting to the ith element
\end{lstlisting}}
\defverbatim[colored]\arrayIndexAccessTwo{
\begin{lstlisting}[language=C,tabsize=2]
i[tab];       // reading from the ith element
i[tab] = ...; // writting to the ith element
\end{lstlisting}}
\defverbatim[colored]\arrayIndexAccessThree{
\begin{lstlisting}[language=C,tabsize=2]
*(tab + i);       // equivalent to *(i + tab) 
*(tab + i) = ...; // writting to the ith element, equivalent to *(i + tab) 
\end{lstlisting}}
  	\begin{frame}{Accès aux éléments du tableau}
  		\begin{block}{Syntaxe}
  			Pour accéder au ième élément du tableau :
  			\arrayIndexAccess
  			Une autre syntaxe possible est :
  			\arrayIndexAccessTwo
  			En utilisant l'arithmétique des pointeurs, les expressions ci-dessus sont équivalentes à :
  			\arrayIndexAccessThree
  		\end{block}
  	\end{frame}
	
  	\begin{frame}{Les chaines}
  		\begin{block}{Explication :}
  			Les chaînes sont représentées comme un tableau de caractères en C. Chaque chaîne doit se terminer par le caractère spécial \alert{`\textbackslash0'} également appelé le caractère nul. Pour les chaînes déclarées statiquement\footnote[frame]{Au moment de la compilation}, le compilateur les ajoute implicitement.
  		\end{block}
  		\begin{alertblock}{ATTENTION :}
  			Le type \texttt{String} \alert{n'existe pas} en C !
  		\end{alertblock}
  	\end{frame}
  
\defverbatim[colored]\Strings{
\begin{lstlisting}[language=C,tabsize=2]
char str[] = "Hello";
\end{lstlisting}}
\defverbatim[colored]\StringsArr{
\begin{lstlisting}[language=C,tabsize=2]
char str[] = {'H', 'e', 'l', 'l', 'o', '\0'};
\end{lstlisting}}

  	\begin{frame}{Les chaines}
  		\framesubtitle{Example :}
  		Soit le code suivant :
  		\Strings
  		Equivalent à :
  		\StringsArr
  		\begin{figure}[!h]
  			\centering
  			\begin{tikzpicture} [nodes in empty cells,
  				nodes={minimum width=1cm, minimum height=1cm},
  				row sep=-\pgflinewidth, column sep=-\pgflinewidth]
  				% border/.style={draw}
  				\matrix(vector)[matrix of nodes, ampersand replacement=\&, % <- added ampersand replacement
  				row 1/.style={nodes={draw=none, minimum width=1cm, fill=fibeamer@black}},
  				nodes={freestruct, anchor=center}]
  				{ % use \& instead of & as column separator
  					{0} \& {1} \& {2} \& {3} \& {4} \& {5}\\
  					H \& e \& l \& l \& o \& \textbackslash0\\
  				};
  				
  				\draw (-3,-2.2) node{str};
  				\draw[-{Latex[length=2.25mm]}] (-3,-2) -- (-3,-1);
  				
  				\draw (0,-3) node{Figure - Représentation mémoire du chaine str};
  			\end{tikzpicture}
  		\end{figure}
  	\end{frame}
  
  	\begin{frame}{Les chaines}
  		\begin{alertblock}{Quelques remarques :}
  			\begin{itemize}
  				\item Etant donné que les chaînes sont intrinsèquement des tableaux, le problème de «désintégration du pointeur» est toujours présent.
  				\item Le caractère nul à la fin est utilisé pour indiquer la fin de la chaîne, par conséquent, lorsqu'elle n'est pas présente, les fonctions standard appelées sur la chaîne accéderont aux éléments au-delà de la limite du tableau jusqu'à ce qu'ils rencontrent 0 quelque part en mémoire.
  				\item Pour obtenir la longueur d'une chaîne \alert{terminée par le caractère nul}, la fonction \texttt{strlen} définie dans l'en-tête \texttt{string.h} peut être utilisée.
  			\end{itemize}
  		\end{alertblock}
  	\end{frame}
  	
  	%%%%%% LA MEMOIRE %%%%%%
  	\subsection{La mémoire}
  	\begin{frame}{La mémoire}
  		\framesubtitle{Disposition de la mémoire d'un programme}
  		\begin{figure}[!h]
  			\centering
  			\begin{tikzpicture}[scale=0.8]
  				\cell{\texttt{OS / Kernel}}  \cellcom{Adresses hautes}
  				\separator
  				\cell{\texttt{Pile}} \coordinate (O1) at (currentcell.center);
  				\padding{2}{} \coordinate (O2) at (currentcell.center);
  				\cell{\texttt{Tas}}  \coordinate (O3) at (currentcell.center);
  				\separator
  				\cell{\texttt{BSS}}  
  				\separator\separator
  				\cell{\texttt{Data}} 
  				\separator
  				\cell{\texttt{Text}} \cellcom{Adresses basses}
  				\cell[draw=none]{L'image mémoire d'un processus (Adresse virtuelle)}
  				\draw[-{Latex[length=4mm]}, ultra thick] ([shift={(1.2,-0.2)}]O1) -- ([shift={(1.2,0.3)}]O2);
  				\draw[-{Latex[length=4mm]}, ultra thick] ([shift={(1.2,0.2)}]O3) -- ([shift={(1.2,-0.3)}]O2);
  				
  				\draw[-{Latex[length=4mm]}, ultra thick] ([shift={(-1.2,-0.2)}]O1) -- ([shift={(-1.2,0.3)}]O2);
  				\draw[-{Latex[length=4mm]}, ultra thick] ([shift={(-1.2,0.2)}]O3) -- ([shift={(-1.2,-0.3)}]O2);
  				
  				\draw[-{Latex[length=4mm]}, ultra thick] ([shift={(0,-0.2)}]O1) -- ([shift={(0,0.3)}]O2);
  				\draw[-{Latex[length=4mm]}, ultra thick] ([shift={(0,0.2)}]O3) -- ([shift={(0,-0.3)}]O2);
  			\end{tikzpicture}
  		\end{figure}
  	\end{frame}
  	
  	\begin{frame}{La mémoire}
  		\framesubtitle{Les différents segments mémoire}
  		\begin{itemize}
  			\item \alert{Pile} : La pile est une région de mémoire (structure LIFO) réservée aux variables locales, à l'environnement de fonctions..
  			\item \alert{Tas} : Le tas est le segment réservé à l'allocation mémoire demandée par le programmeur pour des variables dont la taille ne peut être connue qu'au moment de l'exécution.
  			\item \alert{BSS}\footnote[frame]{également appelé "\alert{segment de données non initialisé}".} : Les données de ce segment sont initialisées par le kernel à 0 avant que le programme commence à s'exécuter. En générale, ce segment contient toutes les variables \alert{globales} et \alert{statiques}\footnote[frame]{déclaré avec le mot-clé \texttt{static}} qui sont \alert{initialisées} à zéro ou qui \alert{n'ont pas d'initialisation} explicite dans le code source. le segment BSS est \alert{Read-Write}.
  		\end{itemize}
  	\end{frame}
  
  	\begin{frame}{La mémoire}
  		\framesubtitle{Les différents segments}
  		\begin{itemize}
  			\item \alert{Data} : Le segment de données ou le segment de données initialisé. Cette partie de l'espace d'adressage virtuel d'un programme contient les variables \alert{globales} et \alert{statiques}\footnote[frame]{déclaré avec le mot-clé \texttt{static}} qui sont \alert{initialisées} par le \alert{programmeur}. Ce segment peut être encore classé en deux zones :
  			\begin{itemize}
  				\item Zone contenant des données initialisées en lecture seule (RoData).
  				\item Zone contenant des données initialisées en lecture-écriture.
  			\end{itemize}
  			\item \alert{Text} : Le segment de texte, également appelé segment de code, est la section de la mémoire qui contient les instructions exécutables d'un programme.
  		\end{itemize}
  	\end{frame}

\defverbatim[colored]\dataDemoOne{
\begin{lstlisting}[language=C,tabsize=2]
// includes ..
char str1[] = "Hello";
const char* str2 = "World";

int main() {
	str1[0] = 'A';
	str2[0] = 'B';
	puts(str1);
	puts(str2);
}
\end{lstlisting}}
\defverbatim[colored]\dataDemoOneSolution{
\begin{lstlisting}[language=C,tabsize=2]
// includes ..
char str1[] = "Hello";
const char* str2 = "World";

int main() {
	str1[0] = 'A';
	str2[0] = 'B'; // Compilation error: str2 is declared const
	puts(str1);
	puts(str2);
}
\end{lstlisting}}

\defverbatim[colored]\dataDemoTwo{
\begin{lstlisting}[language=C,tabsize=2]
// includes ..
char str1[] = "Hello";
char* str2 = "World";

int main() {
	str1[0] = 'A';
	str2[0] = 'B';
	puts(str1);
	puts(str2);
}
\end{lstlisting}}
\defverbatim[colored]\dataDemoTwoSolution{
\begin{lstlisting}[language=C,tabsize=2]
// includes ..
char str1[] = "Hello"; // Lives in the RW Data segment
char* str2 = "World";  // "World" Lives in the Ro Data segment
                       // str2 lives in the RW Data segment
		
int main() {
	str1[0] = 'A';
	str2[0] = 'B'; // Segfault here
	puts(str1);
	puts(str2);
}
\end{lstlisting}}

\defverbatim[colored]\dataDemoThree{
\begin{lstlisting}[language=C,tabsize=2]
// includes ..

int main() {
	char str1[] = "Hello";
	char* str2 = "World";
	str1[0] = 'A';
	str2[0] = 'B';
	puts(str1);
	puts(str2);
}
\end{lstlisting}}
\defverbatim[colored]\dataDemoThreeSolution{
\begin{lstlisting}[language=C,tabsize=2]
// includes ..
		
int main() {
	char str1[] = "Hello"; // Lives on the stack no problem
	char* str2 = "World";  // "World" Lives in the Ro Data segment
	                       // str2 lives on the stack
	str1[0] = 'A';
	str2[0] = 'B'; // Segfault here
	puts(str1);
	puts(str2);
}
\end{lstlisting}}


  	\begin{frame}{La mémoire}
  		\framesubtitle{Pouvez-vous prédire la sortie de ce programme?}
  		Example 1:
  		\dataDemoOne
  	\end{frame}
  	\begin{frame}{La mémoire}
  		\framesubtitle{Ca ne compile même pas}
  		Solution 1:
  		\dataDemoOneSolution
  	\end{frame}
  
  	\begin{frame}{La mémoire}
  		\framesubtitle{Pouvez-vous prédire la sortie de ce programme?}
  		Example 2:
  		\dataDemoTwo
  	\end{frame}
  	\begin{frame}{La mémoire}
  		\framesubtitle{Vos prédiction était probablement fausse}
  		Solution 2:
  		\dataDemoTwoSolution
  	\end{frame}
  
  	\begin{frame}{La mémoire}
  		\framesubtitle{Cela devrait fonctionner, non?}
  		Example 3:
  		\dataDemoThree
  	\end{frame}
  	
  	\begin{frame}{La mémoire}
  		\framesubtitle{Nope, raté}
  		Solution 3:
  		\dataDemoThreeSolution
  	\end{frame}
   	
  	\begin{frame}{La mémoire}
  		\framesubtitle{Mémoire statique vs Mémoire dynamique}
  		\begin{block}{Mémoire statique}
  			Une mémoire est appelée \alert{statique } lorsque sa taille est déterminée lors de la \alert{compilation}. Ce type de mémoire est généralement alloué sur la \alert{pile} (stack).
  		\end{block}
  		\begin{block}{Mémoire dynamique}
  			Une mémoire est dite \alert{dynamique} lorsque sa taille est déterminée pendant le temps \alert{d'exécution}. Ce type de mémoire est généralement alloué à partir du \alert{tas} (heap) via un appel système (syscall).
  		\end{block}
  	\end{frame} 	

  	\begin{frame}{La mémoire}
  		\framesubtitle{Mémoire statique vs Mémoire dynamique}
  		\begin{alertblock}{
  			Chaque mémoire allouée à partir du tas doit être libérée à un moment donné pendant l'exécution du programme. Pour chaque appel à \texttt{malloc} ou \texttt{calloc}, il doit nécessairement y avoir un appel correspondant à \texttt{free}.}
  		\end{alertblock}
  		\begin{block}{La fonction \texttt{free}}
  			Pour libérer de la mémoire, la fonction \texttt{free} doit être utilisée :
  			\alert{Profile} : \texttt{void free(void* ptr);} \\
  			Exemple : \texttt{free(tab);}
  		\end{block}
  		\begin{alertblock}{Attention, seule la mémoire qui existe dans le tas doit être libérée.}
  		\end{alertblock}
  	\end{frame}
  
\defverbatim[colored]\leakExmplOne{
\begin{lstlisting}[language=C,tabsize=2]
struct Point 
{
	float x, y;
};	

int main() 
{
	struct Point** hexagone = malloc(6 * sizeof(struct Point*));
	
	for (int i = 0; i < 6; i++) {
		hexagone[i] = malloc(sizeof(struct Point));
		// Init of hexagone[i]  ...
	}

	free(hexagone);
}
\end{lstlisting}}

\defverbatim[colored]\leakExmplTwo{
\begin{lstlisting}[language=C,tabsize=2]	
free(hexagone); // memory leak here !
\end{lstlisting}}

\defverbatim[colored]\leakExmplSolution{
\begin{lstlisting}[language=C,tabsize=2]	
for (int i = 0; i < 6; i++) {
	free(hexagone[i]); // freach each point 
}

free(hexagone); // free the array of pointers
\end{lstlisting}}

  	\begin{frame}{La mémoire}
  		\leakExmplOne
  	\end{frame}
  
  	\begin{frame}{La mémoire : Fuite mémoire}
  		\begin{figure}[!h]
  			\centering
  			\begin{tikzpicture} [nodes in empty cells,
  				nodes={minimum width=1cm, minimum height=1cm},
  				row sep=-\pgflinewidth, column sep=-\pgflinewidth]
  				% border/.style={draw}
  				\matrix(vector)[matrix anchor=center, xshift=0.5cm, matrix of nodes, ampersand replacement=\&, % <- added ampersand replacement
  				row 1/.style={nodes={draw=none, minimum width=1cm, fill=fibeamer@black}},
  				nodes={freestruct, anchor=center}]
  				{ % use \& instead of & as column separator
  					{0} \& {1} \& {2} \& {3} \& {4} \& {5} \\
  					$.$ \& $.$ \& $.$ \& $.$ \& $.$ \& $.$ \\
  				};
  				\draw (-3.5,-0.5) node{hexagone :};
				\pgfmathsetmacro{\structdimx}{0.5}
				\pgfmathsetmacro{\structdimy}{0.4}
				
	  			\drawstruct{(-3,-2)}
	  			\structcell{.} \coordinate (p1) at (currentcell.center);
	  			\structcell{x=4} 
	  			\structcell{y=0}
	  			\structname{struct Point}
	  			
	  			\drawstruct{(-1.5,-2)}
	  			\structcell{.} \coordinate (p2) at (currentcell.center);
	  			\structcell{x=2} 
	  			\structcell{y=3.5}
	  			
	  			\drawstruct{(0,-2)}
	  			\structcell{.} \coordinate (p3) at (currentcell.center);
	  			\structcell{x=-2} 
	  			\structcell{y=3.5}
	  			
	  			\drawstruct{(1.5,-2)}
	  			\structcell{.} \coordinate (p4) at (currentcell.center);
	  			\structcell{x=-2} 
	  			\structcell{y=0}
	  			
	  			\drawstruct{(3,-2)}
	  			\structcell{.} \coordinate (p5) at (currentcell.center);
	  			\structcell{x=-2} 
	  			\structcell{y=-3.5}
	  			
	  			\drawstruct{(4.5,-2)}
	  			\structcell{.} \coordinate (p6) at (currentcell.center);
	  			\structcell{x=4} 
	  			\structcell{y=0}
	  			
	  			\pgfmathsetmacro{\structdimx}{1.6}
	  			\pgfmathsetmacro{\structdimy}{0.5}
  				\draw[-{Latex[length=2.4mm]}, thick] (-2,-0.5) -- (p1);
  				\draw[-{Latex[length=2.4mm]}, thick] (-1,-0.5) -- (p2);
  				\draw[-{Latex[length=2.4mm]}, thick] (0,-0.5) -- (p3);
  				\draw[-{Latex[length=2.4mm]}, thick] (1,-0.5) -- (p4);
  				\draw[-{Latex[length=2.4mm]}, thick] (2,-0.5) -- (p5);
  				\draw[-{Latex[length=2.4mm]}, thick] (3,-0.5) -- (p6);
  			\end{tikzpicture}
  		\end{figure}
  	\end{frame}
  
  	\begin{frame}{La mémoire : Fuite mémoire}
  		\leakExmplTwo
  		\begin{figure}[!h]
  			\centering
  			\begin{tikzpicture} [nodes in empty cells,
  				nodes={minimum width=1cm, minimum height=1cm},
  				row sep=-\pgflinewidth, column sep=-\pgflinewidth]
  				% border/.style={draw}
  				\matrix(vector)[matrix anchor=center, xshift=0.75cm, matrix of nodes, ampersand replacement=\&, % <- added ampersand replacement
  				row 1/.style={nodes={draw=none, minimum width=1cm, fill=fibeamer@black}},
  				nodes={freestruct, anchor=center}]
  				{ % use \& instead of & as column separator
  					{0} \& {1} \& {2} \& {3} \& {4} \& {5} \\
  					$.$ \& $.$ \& $.$ \& $.$ \& $.$ \& $.$ \\
  				};
  				\draw (-3.25,-0.5) node{hexagone :};
  				\pgfmathsetmacro{\structdimx}{0.5}
  				\pgfmathsetmacro{\structdimy}{0.4}
  				
  				\drawstruct{(-3,-1.5)}
  				\structcell{.} \coordinate (p1) at (currentcell.center);
  				\structcell{x=4} 
  				\structcell{y=0}
  				\structname{struct Point}
  				
  				\drawstruct{(-1.5,-1.5)}
  				\structcell{.} \coordinate (p2) at (currentcell.center);
  				\structcell{x=2} 
  				\structcell{y=3.5}
  				
  				\drawstruct{(0,-1.5)}
  				\structcell{.} \coordinate (p3) at (currentcell.center);
  				\structcell{x=-2} 
  				\structcell{y=3.5}
  				
  				\drawstruct{(1.5,-1.5)}
  				\structcell{.} \coordinate (p4) at (currentcell.center);
  				\structcell{x=-2} 
  				\structcell{y=0}
  				
  				\drawstruct{(3,-1.5)}
  				\structcell{.} \coordinate (p5) at (currentcell.center);
  				\structcell{x=-2} 
  				\structcell{y=-3.5}
  				
  				\drawstruct{(4.5,-1.5)}
  				\structcell{.} \coordinate (p6) at (currentcell.center);
  				\structcell{x=4} 
  				\structcell{y=0}
  				
  				\pgfmathsetmacro{\structdimx}{1.6}
  				\pgfmathsetmacro{\structdimy}{0.5}
  				\draw[fibeamer@lightRed, ultra thick] (-2.75,-1.5) to (4.25,0.5);
  				\draw[fibeamer@lightRed, ultra thick] (-2.75,0.5) to (4.25,-1.5);
  			\end{tikzpicture}
  		\end{figure}
  	\end{frame}
  
  	\begin{frame}{La mémoire : Fuite mémoire}
  		\begin{block}{Explication}
  			Ce que nous venons de voir est un exemple de \alert{fuite de mémoire}. \\
  			Une fuite de mémoire se produit lorsqu'une mémoire allouée dynamiquement n'est jamais libérée.
  		\end{block}
  		Pour résoudre le problème, nous devons procéder comme suit :
  		\leakExmplSolution
  	\end{frame}
  
  	\begin{frame}{La mémoire}
  		\framesubtitle{Mémoire statique vs Mémoire dynamique}
  		Avantages:
		\begin{itemize}
			\item Il n'y a pas de coût d'allocation.
			\item Cache local la plupart du temps car il est situé dans la pile.
		\end{itemize}
		Inconvénients:
		\begin{itemize}
			\item Très local, en raison de la nature de la pile.
			\item Taille limitée.
			\item La taille doit être fixée pendant la compilation\footnote[frame]{C autorise l'allocation de mémoire sur la pile dont la taille est déterminée lors de l'exécution, ceci est interdit en C++.}.
		\end{itemize}
  	\end{frame}
  
  	\begin{frame}{La mémoire}
  		\framesubtitle{Mémoire statique vs Mémoire dynamique}
  		Avantages:
  		\begin{itemize}
  			\item Flexible, la taille peut être déterminée au moment de l'exécution.
  			\item Globale.
  			\item Peut gérer des tailles que la pile ne peut pas gérer.
  		\end{itemize}
  		Inconvénients:
	  	\begin{itemize}
  			\item L'allocation peut être très coûteuse car elle nécessite un passage du mode utilisateur au mode noyau.
			\item Responsabilité de libérer la mémoire à la fin de l'utilisation.
	  	\end{itemize}
  	\end{frame}


\defverbatim[colored]\mallocExample{
\begin{lstlisting}[language=C,tabsize=2]
void* p1 = malloc(256); // 256 bytes are allocated
int* p2 = (int*)malloc(4 * sizeof(int)); // 4 * sizeof(int) bytes are allocated
struct A* p3 = (struct A*)malloc(2 * sizeof(struct A)); // 2 * sizeof(struct A) bytes are allocated
p2[0]; // Access to uninitialized memory !
p3[0].a; // Access to uninitialized memory !
\end{lstlisting}}
  	\begin{frame}{La mémoire}
  		\framesubtitle{malloc, calloc et realloc}
  		\begin{block}{malloc}
  			\alert{Profile} : \texttt{void* malloc(size\_t size);} \\
  			Alloue ce qui lui est passé comme argument en octets mais \alert{n'effectue aucune initialisation}.
  		\end{block}
  		\begin{exampleblock}{Exemple :}
  			\mallocExample
  		\end{exampleblock}
    \end{frame}

\defverbatim[colored]\callocExample{
\begin{lstlisting}[language=C,tabsize=2]
void* p1 = calloc(1, 256); // 256 bytes are allocated
int* p2 = (int*)calloc(4, sizeof(int)); // 4 * sizeof(int) bytes are allocated
struct A* p3 = (struct A*)calloc(2, sizeof(struct A)); // 2 * sizeof(struct A) bytes are allocated
assert(p2[0] == 0); // true
assert(p3[0].a == 0); // true
assert(p3[1].b == 0); // true
assert(p3[1].str[0] == 0); // true
\end{lstlisting}}
	\begin{frame}{La mémoire}
		\framesubtitle{malloc, calloc et realloc}
		\begin{block}{calloc}
			\alert{Profile} : \texttt{void* calloc(size\_t nmemb, size\_t size);} \\
			Alloue $nmemb * size$ octets et les \alert{initialise} à zéro.
		\end{block}
		\begin{exampleblock}{Exemple :}
			\callocExample
		\end{exampleblock}
	\end{frame}

\defverbatim[colored]\reallocExampleOne{
\begin{lstlisting}[language=C,tabsize=2]
int* p1 = (int*)calloc(4, sizeof(int)); // 4 * sizeof(int) bytes
p[0] = 1; p[1] = 2; p[2] = 3; p[3] = 4;
int* p2 = (int*)realloc(p1, 6 * sizeof(int));
assert(p2[0] == 1); // true
assert(p2[1] == 2); // true
assert(p2[5] == 6); // Access to uninitialized memory !
assert(p1[5] == 6); // U.B !
\end{lstlisting}}

\defverbatim[colored]\reallocExampleTwo{
\begin{lstlisting}[language=C,tabsize=2]		
int* p3 = (int*)realloc(p2, 3 * sizeof(int));
assert(p3[0] == 1); // true
assert(p3[1] == 2); // true
assert(p3[3] == 3); // U.B!
assert(p2[3] == 3); // U.B!

int* p4 = (int*)realloc(p3, 0); // Equivalent to free(p3)
void* p5 = realloc(NULL, 8); // Equivalent to malloc(8)
void* p6 = realloc(NULL, 0); // Equivalent to malloc(0)
\end{lstlisting}}
	\begin{frame}{La mémoire}
		\framesubtitle{malloc, calloc et realloc}
		\begin{block}{realloc}
			\alert{Profile} : \texttt{void* realloc(void* ptr, size\_t size);} \\
			Change la taille du bloc de mémoire pointé par \texttt{ptr} en \texttt{size} octets. 
			\begin{itemize}
				\item Si \texttt{size} > taille de \texttt{ptr} : La mémoire pointée par le pointeur retourné par \texttt{realloc} sera de taille \texttt{size}. Le contenu de \texttt{ptr} est garanti d'être copié mais la mémoire ajoutée ne sera \alert{pas initialisée}
				\item Si \texttt{size} < taille de \texttt{ptr} : Le contenu de \texttt{ptr} sera copié jusqu'à \texttt{size} octets, le reste du contenu de ptr sera ignoré. La taille de la mémoire pointée par la valeur de retour sera donc \texttt{size}.
				\item Si \texttt{ptr} est \texttt{NULL} : Cela aura le même effet que malloc.
			\end{itemize}
		\end{block}
	\end{frame}

	\begin{frame}{La mémoire}
		\framesubtitle{malloc, calloc et realloc}
		\begin{block}{realloc}
			\begin{itemize}
				\item Si \texttt{size} est $0$ et \texttt{ptr} n'est pas \texttt{NULL} :  Cela aura le même effet que free.
			\end{itemize}
		\end{block}
		\begin{alertblock}{N.B. :}
			- Sauf si \texttt{ptr} est \texttt{NULL}, il doit avoir été renvoyé par un appel antérieur
			à \texttt{malloc()}, \texttt{calloc()} ou \texttt{realloc()}. \\
			- L'accès et/ou l'écriture au pointeur a passé à \texttt{realloc} après l'appel est un \alert{comportement indéfini}\\
		\end{alertblock}
	\end{frame}

	\begin{frame}{La mémoire}
		\framesubtitle{malloc, calloc et realloc}
		\begin{exampleblock}{Exemple :}
			\reallocExampleOne
		\end{exampleblock}
	\end{frame}

	\begin{frame}{La mémoire}
		\framesubtitle{malloc, calloc et realloc}
		\begin{exampleblock}{Exemple :}
			\reallocExampleTwo
		\end{exampleblock}
	\end{frame}

	\begin{frame}{La mémoire}
		\framesubtitle{malloc, calloc et realloc}
		\begin{alertblock}{N.B. :}
			Le standard C ne dit rien quand 0 est passé à \texttt{malloc} (le comportement est spécifique au système d'exploitation).
			\begin{itemize}
				\item Sous Linux : \texttt{malloc(0)} renvoie \texttt{NULL}
				\item Sous Windows : \texttt{malloc(0)} renvoie un pointeur \alert{valide} sur lequel free pourrait être appelé.
			\end{itemize}
		\end{alertblock}
	\end{frame}

	\begin{frame}{La mémoire}
		\framesubtitle{Les pointeurs}
		\begin{block}{Définition}
			Un pointeur est une variable qui contient l'adresse d'une région de mémoire. Un pointeur peut contenir une adresse valide ou non (Exemple : le pointeur \texttt{NULL}). \\
		\end{block}
		Les pointeurs font généralement 4 ou 8 octets en fonction de l'architecture du processeur (32 ou 64 bits): 
		\begin{itemize}
			\item Avec un pointeur 32 bits, nous avons 4 Go de mémoire adressable.
			\item Avec un pointeur 64 bits, nous avons environ 17 milliards de Go de mémoire.
		\end{itemize}
	\end{frame}

\defverbatim[colored]\ptrSyntax{
\begin{lstlisting}[language=C,tabsize=4]	
Typename* ptr_name; // Recommended
Typename *ptr_name; // Recommended
Typename*ptr_name;
Typename * ptr_name;
\end{lstlisting}}

\defverbatim[colored]\ptrAdrSyntax{
\begin{lstlisting}[language=C,tabsize=4]	
Typename var = ...;
Typename* ptr = &var;
\end{lstlisting}}
\defverbatim[colored]\ptrAdrExmpl{
\begin{lstlisting}[language=C,tabsize=4]	
int var = 42;
int* ptr = &var;
\end{lstlisting}}
\defverbatim[colored]\ptrAdrArr{
\begin{lstlisting}[language=C,tabsize=4]	
int arr[] = {1, 2, 3};
int* ptrCpy = arr;
int* ptrArr = &arr;
assert(arr == &arr); // true
assert(ptrCpy == arr); // true
assert(ptrCpy == ptrArr); // true
assert(&ptrCpy == arr); // false
assert(&ptrCpy == &ptrArr); // false
\end{lstlisting}}
	\begin{frame}{La mémoire}
		\framesubtitle{Les pointeurs}
		\begin{block}{Syntaxes possibles}
			Toutes les syntaxes suivantes sont valides. Cependant, les deux premiers sont les plus courants :
			\ptrSyntax
		\end{block}
	\end{frame}
	
	\begin{frame}{La mémoire}
		\framesubtitle{Les pointeurs}
		\begin{block}{Comment obtenir l'adresse d'une variable}
			Pour obtenir l'adresse d'une variable, l'opérateur \alert{\&} doit être utilisé. \\
			Exemple :
			\ptrAdrSyntax
		\end{block}
		\begin{exampleblock}{Example : }
			\ptrAdrExmpl
		\end{exampleblock}
	\end{frame}

	\begin{frame}{La mémoire}
		\framesubtitle{Pointeur et variable sur la pile}
		\begin{figure}[!h]
			\centering
			\begin{tikzpicture}[scale=0.9]
				\stacktop{}
				\separator
				\cell{\texttt{var=42}}  \cellcomL{0xABCDEF00} \coordinate (p1) at (currentcell.east);
				\separator
				\padding{2}{...}
				\separator
				\cell{\texttt{ptr=0xABCDEF00}}  \cellcomL{0xABCDEFBA} \coordinate (p2) at (currentcell.east);
				\separator
				\stackbottom{}
				\cell[draw=none]{La Pile}
				
				\draw[-{Latex[length=3.3mm]}, thick] (p2) .. controls (4,-4) and (4,0) .. (p1);
			\end{tikzpicture}
		\end{figure}
	\end{frame}

	\begin{frame}{La mémoire}
		\framesubtitle{Pointeur sur la pile et variable sur le tas}
		\begin{figure}[!h]
			\centering
			\begin{tikzpicture}[scale=0.8]
				\stacktop{}
				\cell{\texttt{...}} \cellcomL{0xABCDEF00}
				\cell{\texttt{ptr=0xFFFFEF08}} \cellcomL{0xABCDEF04} \coordinate (p1) at (currentcell.east);
				\cell{\texttt{...}} \cellcomL{0xABCDEF08}
				\stackbottom{}
				\cell[draw=none]{La Pile}

				\drawstruct{(5,1)})
				\structcell{...}
				\structcell{...}
				\structcell{...}
				\structcell{var=42} \coordinate (p2) at (currentcell.west);
				\structcell{...}
				\structcell{...}
				\structcell{...}
				\draw (5,-7) node{Tas};
				\draw (8,-3) node{0xFFFFEF08};
				
				\draw[-{Latex[length=3.3mm]}, thick] (p1) -- (p2);
			\end{tikzpicture}
		\end{figure}
	\end{frame}

	\begin{frame}{La mémoire}
		\framesubtitle{Les pointeurs}
		\begin{alertblock}{Attention : }
			Lorsque l'opérateur \alert{\&} est utilisé sur un tableau, il renvoie une adresse vers le premier élément de ce tableau et non un pointeur vers la première adresse du tableau. \\
		\end{alertblock}
		\begin{exampleblock}{Example :}
			\ptrAdrArr
		\end{exampleblock}
	\end{frame}
	
\defverbatim[colored]\ptrDerefSyntax{
\begin{lstlisting}[language=C,tabsize=4]	
TypeName var = ...;
TypeName* ptr = &var;
assert(*ptr == var); // true
*ptr = ...;          // changes the content of var
\end{lstlisting}}

	\begin{frame}{La mémoire}
		\framesubtitle{Les pointeurs}
		\begin{block}{Comment déréférencer un pointeur}
			Déréférencer un pointeur permet d'accéder à la valeur pointée par le pointeur. Pour cela, l'opérateur \alert{\texttt{*}} doit être utilisé. \\
			Example:
			\ptrDerefSyntax
		\end{block}
	\end{frame}

\defverbatim[colored]\ptrStructSyntaxOne{
\begin{lstlisting}[language=C,tabsize=4]	
struct StructName* ptr = &var;
ptr->field1_name = ...;
ptr->field2_name[0] = ...; // supposing that field2_name is an array this will change the first element of field2_name
ptr->field1_name; // accessing field1_name in the struct var
ptr->field2_name[0]; // acessing the first element in field2_name in the struct var
\end{lstlisting}}
\defverbatim[colored]\ptrStructSyntaxTwo{
\begin{lstlisting}[language=C,tabsize=4]	
struct StructName* ptr = &var;
(*ptr).field1_name = ...; // *ptr is between parentheses because "*" have a lower percedance level than "."
(*ptr).field2_name[0] = ...; // supposing that field2_name is an array this will change the first element of field2_name
(*ptr).field1_name; // accessing field1_name in the struct var
(*ptr).field2_name[0]; // acessing the first element in field2_name in the struct var
\end{lstlisting}}
	\begin{frame}{La mémoire}
		\framesubtitle{Les pointeurs}
		\begin{alertblock}{ATTENTION :}
			- Déréférencer un pointeur NULL est un comportement indéfini. \\
			- Déréférencer un pointeur sur lequel \texttt{free()} a été appelé est un comportement indéfini.\\
			- En général, le déréférencement d'un \alert{dangling pointer} entraînera une corruption de pile ou un plantage du programme.
		\end{alertblock}
		\begin{block}{Définition}
			Un \alert{dangling pointer}\footnote[frame]{traduction littérale : « pointeur pendouillant » ou « pointeur sautillant »} est un pointeur pointant vers une adresse mémoire valide qui a été libérée ou détruite.
		\end{block}
	\end{frame}
\defverbatim[colored]\danglinPtrOne{
\begin{lstlisting}[language=C,tabsize=4]	
int main() {
	int* ptr = NULL;
	{
		int x = 42;
		ptr = &x;
	}
	printf("content of ptr : %d\n", *ptr);
	*ptr = 3;
}
\end{lstlisting}}
\defverbatim[colored]\danglinPtrOneSolution{
\begin{lstlisting}[language=C,tabsize=4]	
int main() {
	int* ptr = NULL;
	{
		int x = 42;
		ptr = &x;
	}
	// x is out of scope (it's popped out of the stack)
	printf("content of ptr : %d\n", *ptr); // ptr is now a 
	                                       // dangling pointer
	*ptr = 3; // this will either corrupt the stack or
	          // segfault
}
\end{lstlisting}}

\defverbatim[colored]\danglingPtrPartOne{
\begin{lstlisting}[language=C,tabsize=4]	
int* ptr = NULL;
\end{lstlisting}}

\defverbatim[colored]\danglingPtrPartTwo{
\begin{lstlisting}[language=C,tabsize=4]	
int x = 42;
ptr = &x;
\end{lstlisting}}

\defverbatim[colored]\danglingPtrPartThree{
\begin{lstlisting}[language=C,tabsize=4]
{	
	int x = 42;
	ptr = &x;
} // x is out of scope (it's popped out of the stack)
\end{lstlisting}}

	\begin{frame}{La mémoire}
		\framesubtitle{Les pointeurs}
		Exemple 1 :
		\danglinPtrOne
	\end{frame}

	\begin{frame}{La mémoire}
		\framesubtitle{Dangling pointers}
		Solution 1 :
		\danglinPtrOneSolution
	\end{frame}

	\begin{frame}{La mémoire : Dangling pointers}
		\danglingPtrPartOne
		\begin{figure}[!h]
			\centering
			\begin{tikzpicture}[scale=0.9]
				\stacktop{}
				\separator
				\startframe
				\cell{\texttt{args...}}
				\cell{\texttt{@r}} \cellcom{0x12345600}
				\cell{\texttt{ancien @fp}} \cellcom{0x12345604}
				\cell{\texttt{ptr=0}}  \cellcom{0x12345608}
				\finishframe{stack frame\\de main}
				\separator
				\stackbottom{}
				\cell[draw=none]{La Pile}
			\end{tikzpicture}
		\end{figure}
	\end{frame}
	
	\begin{frame}{La mémoire : Dangling pointers}
		\danglingPtrPartTwo
		\begin{figure}[!h]
			\centering
			\begin{tikzpicture}[scale=0.9]
				\stacktop{}
				\separator
				\cell{\texttt{@r}} \cellcomL{0x12345600}
				\cell{\texttt{@fp}} \cellcomL{0x12345604}
				\cell{\texttt{ptr=0x1234560C}}  \cellcomL{0x12345608} \coordinate (p1) at (currentcell.east);
				\cell{\texttt{x=42}}  \cellcomL{0x1234560C} \coordinate (p2) at (currentcell.east);
				\separator
				\stackbottom{}
				\cell[draw=none]{La Pile}
				
				\draw[-{Latex[length=3.3mm]}, thick] (p1) .. controls (3,-3) and (3,-4) .. (p2);
			\end{tikzpicture}
		\end{figure}
	\end{frame}

	\begin{frame}{La mémoire : Dangling pointers}
		\danglingPtrPartThree
		\begin{figure}[!h]
			\centering
			\begin{tikzpicture}[scale=0.85]
				\stacktop{}
				\separator
				\cell{\texttt{@r}} \cellcomL{0x12345600}
				\cell{\texttt{@fp}} \cellcomL{0x12345604}
				\cell{\texttt{ptr=0x1234560C}}  \cellcomL{0x12345608} \coordinate (p1) at (currentcell.east);
				\bcell{\texttt{???}}  \cellcomL{0x1234560C} \coordinate (p2) at (currentcell.east);
				\separator
				\stackbottom{}
				\cell[draw=none]{La Pile}
				
				\draw[-{Latex[length=3.3mm]}, thick] (p1) .. controls (3,-3) and (3,-4) .. (p2);
			\end{tikzpicture}
		\end{figure}
	\end{frame}

\defverbatim[colored]\danglingPtrTwo{
\begin{lstlisting}[language=C,tabsize=4]	
int main()
{
	int* tab = malloc(4 * sizeof(int));
	// Init tab to 1, 2, 3, 4
	free(tab);
	tab[0] = 3;
}
\end{lstlisting}}

\defverbatim[colored]\danglingPtrTwoSolution{
\begin{lstlisting}[language=C,tabsize=4]
int main()
{
	int* tab = malloc(4 * sizeof(int));
	// Init tab to 1, 2, 3, 4
	free(tab);
	// tab is now dangling
	tab[0] = 3; // will crash
}
\end{lstlisting}}

\defverbatim[colored]\danglingPtrTwoSolutionPartOne{
\begin{lstlisting}[language=C,tabsize=4]
int* tab = malloc(4 * sizeof(int));
// Init tab to 1, 2, 3, 4
\end{lstlisting}}

\defverbatim[colored]\danglingPtrTwoSolutionPartTwo{
\begin{lstlisting}[language=C,tabsize=4]
free(tab);
// tab is now dangling
\end{lstlisting}}


	\begin{frame}{La mémoire}
		\framesubtitle{Les pointeurs}
		Exemple 2 :
		\danglingPtrTwo
	\end{frame}
	
	\begin{frame}{La mémoire}
		\framesubtitle{Les pointeurs : Dangling pointers}
		Solution 2 :
		\danglingPtrTwoSolution
	\end{frame}

	\begin{frame}{La mémoire}
		\danglingPtrTwoSolutionPartOne
		\begin{figure}[!h]
			\centering
			\begin{tikzpicture}[scale=0.8]
				\stacktop{}
				\cell{\texttt{tab=0xFFFFEF08}} \cellcomL{0xABCDEF04} \coordinate (p1) at (currentcell.east);
				\stackbottom{}
				\cell[draw=none]{La Pile}
				
				\drawstruct{(5,1)})
				\structcell{...}
				\structcell{1} \coordinate (p2) at (currentcell.west);
				\structcell{2} 
				\structcell{3}
				\structcell{4}
				\structcell{...}
				\draw (5,-6) node{Tas};
				\draw (8,-1) node{0xFFFFEF08};
				
				\draw[-{Latex[length=3.3mm]}, thick] (p1) -- (p2);
			\end{tikzpicture}
		\end{figure}
	\end{frame}
	
	\begin{frame}{La mémoire}
		\danglingPtrTwoSolutionPartTwo
		\begin{figure}[!h]
			\centering
			\begin{tikzpicture}[scale=0.8]
				\stacktop{}
				\cell{\texttt{tab=0xFFFFEF08}} \cellcomL{0xABCDEF04} \coordinate (p1) at (currentcell.east);
				\stackbottom{}
				\cell[draw=none]{La Pile}
				
				\drawstruct{(5,1)})
				\structcell{...}
				\structcell{??} \coordinate (p2) at (currentcell.west);
				\structcell{??} 
				\structcell{??}
				\structcell{??}
				\structcell{...}
				\draw (5,-6) node{Tas};
				\draw (8,-1) node{0xFFFFEF08};
				
				\draw[-{Latex[length=3.3mm]}, thick] (p1) -- (p2);
			\end{tikzpicture}
		\end{figure}
	\end{frame}

\defverbatim[colored]\danglingPtrTwoSolutionTemp{
\begin{lstlisting}[language=C,tabsize=4]
free(tab);
tab = NULL; // check if tab is NULL before we access it later
\end{lstlisting}}

\defverbatim[colored]\danglingPtrTwoSolution{
\begin{lstlisting}[language=C,tabsize=4]
void myfree(void** pptr)
{
	if (pptr && *pptr) {
		free(*pptr);
		*pptr = NULL;
	}
}
\end{lstlisting}}

	\begin{frame}{La mémoire}
		\framesubtitle{Dangling pointers : Solution Possibles}
		\begin{block}{Solution 1 :}
			Une façon de résoudre ce problème consiste à affecter le pointeur à NULL après l'avoir libéré.
			\danglingPtrTwoSolutionTemp
		\end{block}
		\begin{alertblock}{Problème avec la solution ci-dessus}
			Cette solution n'est pas parfaite si on libére \texttt{tab} dans une autre fonction et on l'affecte à \texttt{NULL}, seule la copie locale du pointeur sera \texttt{NULL}.
		\end{alertblock}
	\end{frame}

	\begin{frame}{La mémoire}
		\framesubtitle{Dangling pointers : Solution Possibles}
		\begin{block}{Solution 2 :}
			Une meilleure façon de résoudre ce problème est de passer un pointeur vers le pointeur qui doit être libéré.
			\danglingPtrTwoSolution
		\end{block}
		\begin{alertblock}{Cette solution n'est pas parfaite non plus}
		\end{alertblock}
	\end{frame}
	
	\begin{frame}{La mémoire}
		\framesubtitle{Les pointeurs}
		\begin{block}{Accès aux membres d'un struct/union}
			Il existe deux façons d'accéder à un champ struct via un pointeur :\\ 
			\ptrStructSyntaxOne
		\end{block}
	\end{frame}

	\begin{frame}{La mémoire}
		\framesubtitle{Les pointeurs}
		\begin{block}{Accès aux membres d'un struct/union}
			La deuxième façon est verbeuse et donc déconseillée :\\
			\ptrStructSyntaxTwo
		\end{block}
	\end{frame}
	
\defverbatim[colored]\ptrArithExmp{
\begin{lstlisting}[language=C,tabsize=4]		
int* ptr = ...;
ptr1 = ptr + 2; // will add '8' bytes to the current address. 
				// *ptr1 is equivalent to ptr[2]
ptr1++; // will add '4' bytes to ptr1. 
		// *ptr1 is equivalent to ptr[3]
ptr2 = ptr1 - 3; // will substract '12' bytes from ptr1. 
				 // *ptr2 equivalent to ptr1[-3] or ptr[0]
\end{lstlisting}}
	\begin{frame}{La mémoire}
		\framesubtitle{Arithmétique des pointeurs}
		\begin{block}{Explication}
			Un pointeur n'est qu'une adresse mémoire. Cette adresse est une valeur numérique. Par conséquent, on peut effectuer des opérations arithmétiques sur un pointeur comme on peut le faire sur des valeurs numériques.\\ 
			Il existe quatre opérateurs arithmétiques qui peuvent être utilisés sur les pointeurs: \alert{++}, \alert{-{}-}, \alert{+} et \alert{-}.
		\end{block}
		\begin{block}{La formule}
			Pour un pointeur \alert{\texttt{ptr}} avec le type \alert{\texttt{Typename}}, l'expression \alert{\texttt{ptr+step}} ajoutera \alert{\texttt{step*sizeof(Typename)}} octets au pointeur \alert{\texttt{ptr}}.
		\end{block}
	\end{frame}
  	
  	\begin{frame}{La mémoire}
  		\framesubtitle{Arithmétique des pointeurs}
  		\begin{exampleblock}{Exemple}
  			\ptrArithExmp
  		\end{exampleblock}
  	
  		\begin{alertblock}{Lorsque vous utilisez l'arithmétique du pointeur, veillez à ne pas dépasser la taille allouée.}
  		\end{alertblock}
  	\end{frame}

\defverbatim[colored]\funcPtrSyntax{	
\begin{lstlisting}[language=C,tabsize=4]
ReturnType (*funcPtrName)(ArgType1, ArgType2, ..., ArgTypeN);		
\end{lstlisting}}

  	\begin{frame}{Pointeurs de fonction}
  		\begin{block}{Explication :}
  			Comme les variables, les fonctions existent également en mémoire, nous pouvons donc avoir des pointeurs vers elles. \\
  			Cela peut être utile pour déterminer la fonction à exécuter au moment de l'exécution. On peut imiter le comportement des fonctions virtuelles fournies dans des langages tels que Java ou C++ en utilisant cette fonctionnalité.\\
  			\alert{Syntaxe :}
  			\funcPtrSyntax
  		\end{block}
  	\end{frame}
  	
  	
\defverbatim[colored]\funcPtrExmpl{	
\begin{lstlisting}[language=C,tabsize=4]
// includes ...	
int max(int a, int b) { return a > b ? a : b; }
int min(int a, int b) { return a < b ? a : b; }
typedef int (*MyFuncType)(int, int); // MyFunc is a function
									 // pointer type
int main() {
  	int a = 5, b = 6;
  	srand(time(NULL));
  	int (*func)(int, int) = max; // Equivalent to 
  								 // MyFuncType func = max;
  	if (rand()%2) {
  		func = &min; // same as func = main
  	}
  	return func(a, b);
}
\end{lstlisting}}

  	\begin{frame}{Pointeurs de fonction}  	
  		\begin{exampleblock}{Exemple :}
  			\funcPtrExmpl
  		\end{exampleblock}
  	\end{frame}
  	
  	\begin{frame}{Pointer casting}
  		Comme nous l'avons vu précédemment dans la partie Arithmétique des pointeur, le type de pointeur n'est utilisé que pour déterminer comment la mémoire est accédée et interprétée. Ainsi, nous pouvons convertir un pointeur d'un type à un autre. Cependant, des précautions doivent être prises lors de cette opération, car différents types ont des exigences d'alignement différentes. 
  		D'une manière générale, la conversion d'un type qui a une grande taille à un objet qui a une taille plus petite est acceptable 
  		(par exemple la conversion de \texttt{int*} en \texttt{char*}). \\
  		La conversion vers un type qui a une exigence d'alignement stricte à partir d'un type qui a une exigence d'alignement moins stricte est \alert{interdite}. Au mieux, cela peut entraîner une dégradation des performances ou des valeurs erronées. Au pire, il peut planter tout le programme.
  	\end{frame}
  
\defverbatim[colored]\ptrCasting{	  	
\begin{lstlisting}[language=C,tabsize=4]
A* ptr1 = ....;
B* ptr2 = (B*)ptr1;
\end{lstlisting}}


\defverbatim[colored]\ptrCastingExmpl{	  	
\begin{lstlisting}[language=C,tabsize=4]
unsigned int a = 3721182122; 
	// 3721182122 =  0xDD << 24 | 0xCC << 16 | 0xBB << 8 | 0xAA;
unsigned int* ptrA = &a;
char* bytes = (char*)ptrA;
\end{lstlisting}}
  	\begin{frame}{Pointer casting}
  		\begin{block}{Syntaxe : }
  			Le cast d'un pointeur du type A au type B peut se faire de cette façon :
  			\ptrCasting
  		\end{block}
  		\begin{exampleblock}{Exemple : }
  			\ptrCastingExmpl
  		\end{exampleblock}
  	\end{frame}
  
  	\begin{frame}{Pointer casting}
  		\framesubtitle{Exemple : }
  		\begin{figure}[!h]
  			\centering
  			\begin{tikzpicture} [nodes in empty cells,
  				nodes={minimum width=1cm, minimum height=1cm},
  				row sep=-\pgflinewidth, column sep=-\pgflinewidth]
  				% border/.style={draw}
  				\matrix(vector)[matrix of nodes, ampersand replacement=\&,name=ptrA,
  				row 1/.style={nodes={draw=none, minimum width=1cm, fill=fibeamer@black}},
  				nodes={freestruct, anchor=center}, execute at empty cell={\node[draw=none]{};}]
  				{ % use \& instead of & as column separator
  					{0}  \& {1}\& {2}\& {3} \\
  					% 0xDD \& CC \& BB \& AA  \\
  				};
  			
  				\pgfmathsetmacro{\structdimx}{2.06}
  				\pgfmathsetmacro{\structdimy}{0.4}
  				\drawstruct{(0,0)}
  				\structcell{0xDDCCBBAA};
  				\pgfmathsetmacro{\structdimx}{1.6}
  				\pgfmathsetmacro{\structdimy}{0.5}
  				
  				\draw (-3,-1) node{ptrA};
  				\draw[-{Latex[length=2.25mm]}] (-3,-1.2) .. controls (-3,-2.3) and (-0.5,-2.3) .. (0,-1.4);
  			\end{tikzpicture}
  			\caption{Représentation mémoire de la variable \alert{\texttt{a}}, tel qu'il est interprété par le pointeur \alert{\texttt{ptrA}}}
  		\end{figure}
  	\end{frame}
  	
  	\begin{frame}{Pointer casting}
  		\framesubtitle{Exemple : }
  		\begin{figure}[!h]
  			\centering
  			\begin{tikzpicture} [nodes in empty cells,
  				nodes={minimum width=1cm, minimum height=1cm},
  				row sep=-\pgflinewidth, column sep=-\pgflinewidth]
  				% border/.style={draw}
  				\matrix(vector)[matrix of nodes, ampersand replacement=\&, % <- added ampersand replacement
  				row 1/.style={nodes={draw=none, minimum width=1cm, fill=fibeamer@black}},
  				nodes={freestruct, anchor=center}]
  				{ % use \& instead of & as column separator
  					{0}  \& {1}  \& {2}  \& {3}  \\
  					0xAA \& 0xBB \& 0xCC \& 0xDD \\
  				};
  				
  				\draw (-3,-0.5) node{bytes};
  				\draw[-{Latex[length=2.25mm]}] (-3,-0.65) .. controls (-3,-1.5) and (-2,-1.5) .. (-1.55,-1);
  			\end{tikzpicture}
  			\caption{Représentation mémoire de la variable \alert{\texttt{a}}, tel qu'il est interprété par le pointeur \alert{\texttt{bytes}} (cas d'une machine petit-boutienne)}
  		\end{figure}
  	\end{frame}
  
  	\begin{frame}{Pointer casting}
  		\framesubtitle{Exemple : }
  		\begin{figure}[!h]
  			\centering
  			\begin{tikzpicture} [nodes in empty cells,
  				nodes={minimum width=1cm, minimum height=1cm},
  				row sep=-\pgflinewidth, column sep=-\pgflinewidth]
  				% border/.style={draw}
  				
  				
	  			\matrix (m1) at (0,2.5) (vector)[matrix of nodes, ampersand replacement=\&, % <- added ampersand replacement
	  			row 1/.style={nodes={draw=none, minimum width=1cm, fill=fibeamer@black}},
	  			nodes={freestruct, anchor=center}]
	  			{
	  				{0}  \& {1}  \& {2}  \& {3}  \\
	  				0xAA \& 0xBB \& 0xCC \& 0xDD \\
	  			};
  				
  				\draw (-3,2) node{bytes};
  				\draw[-{Latex[length=2.25mm]}] (-3,2.2) .. controls (-3,3) and (-2,3.2) .. (-1.55,2.5);
  				
  				\matrix(vector)[matrix of nodes, ampersand replacement=\&,name=ptrA,
  				nodes={freestruct, anchor=center}, execute at empty cell={\node[draw=none]{};}]
  				{
  					0xDD \& CC \& BB \& AA  \\
  				};
  				
  				\pgfmathsetmacro{\structdimx}{2.06}
  				\pgfmathsetmacro{\structdimy}{0.4}
  				\drawstruct{(0,0)}
  				\structcell{0xDDCCBBAA};
  				\pgfmathsetmacro{\structdimx}{1.6}
  				\pgfmathsetmacro{\structdimy}{0.5}
  				
  				\draw (-3,-0.5) node{ptrA};
  				\draw[-{Latex[length=2.25mm]}] (-3,-0.7) .. controls (-3,-2) and (-1,-2) .. (0,-1.4);
  				
  				\draw[-] (-1.5,1.5) -- (1.5,0.5);
  				\draw[-] (-0.5,1.5) -- (0.5,0.5);
  				\draw[-] (1.5,1.5) -- (-1.5,0.5);
  				\draw[-] (0.5,1.5) -- (-0.5,0.5);
  			\end{tikzpicture}
  			\caption{La représentation de \alert{\texttt{bytes}} dans le cas d'une machine petit-boutienne}
  		\end{figure}
  	\end{frame}
  
  	\begin{frame}{Pointer casting}
  		\framesubtitle{Exemple : }
  		\begin{figure}[!h]
  			\centering
  			\begin{tikzpicture} [nodes in empty cells,
  				nodes={minimum width=1cm, minimum height=1cm},
  				row sep=-\pgflinewidth, column sep=-\pgflinewidth]
  				% border/.style={draw}
  				
  				\matrix (m1) at (0,2.5) (vector)[matrix of nodes, ampersand replacement=\&, % <- added ampersand replacement
  				row 1/.style={nodes={draw=none, minimum width=1cm, fill=fibeamer@black}},
  				nodes={freestruct, anchor=center}]
  				{
  					{0}  \& {1}  \& {2}  \& {3}  \\
  					0xDD \& 0xCC \& 0xBB \& 0xAA \\
  				};
  				
  				\draw (-3,2) node{bytes};
  				\draw[-{Latex[length=2.25mm]}] (-3,2.2) .. controls (-3,3) and (-2,3.2) .. (-1.55,2.5);
  				
  				\matrix(vector)[matrix of nodes, ampersand replacement=\&,name=ptrA,
  				nodes={freestruct, anchor=center}, execute at empty cell={\node[draw=none]{};}]
  				{
  					0xDD \& CC \& BB \& AA  \\
  				};
  				
  				\pgfmathsetmacro{\structdimx}{2.06}
  				\pgfmathsetmacro{\structdimy}{0.4}
  				\drawstruct{(0,0)}
  				\structcell{0xDDCCBBAA};
  				\pgfmathsetmacro{\structdimx}{1.6}
  				\pgfmathsetmacro{\structdimy}{0.5}
  				
  				\draw (-3,-0.5) node{ptrA};
  				\draw[-{Latex[length=2.25mm]}] (-3,-0.7) .. controls (-3,-2) and (-1,-2) .. (0,-1.4);
  				
  				\draw[-] (-1.5,1.5) -- (-1.5,0.5);
  				\draw[-] (-0.5,1.5) -- (-0.5,0.5);
  				\draw[-] (0.5,1.5) -- (0.5,0.5);
  				\draw[-] (1.5,1.5) -- (1.5,0.5);
  			\end{tikzpicture}
  			\caption{La représentation de \alert{\texttt{bytes}} dans le cas d'une machine gros-boutienne}
  		\end{figure}
  	\end{frame}
  	
  	\begin{frame}{Le Boutisme}
  		\begin{figure}[!h]
  			\centering
  			\begin{tikzpicture}[overlay,remember picture]
  				\node[anchor=south west,xshift=14pt,yshift=50pt]
  				at (current page.south west) {
  					\includegraphics[width=116mm]{resources/endianness_had}
  				};
  				\draw (0,-3.1) node{Figure - Schéma qui résume le boutisme};
  			\end{tikzpicture}
  		\end{figure}
  	\end{frame}
  	
\defverbatim[colored]\voidPtrExmpl{	
\begin{lstlisting}[language=C,tabsize=4]
unsigned int* ptrA = ...;
void* ptr2 = (void*)ptrA;

ptr2 + 1; // Invalid
ptr2 + 2; // Invalid
ptr[1];   // Invalid

int* ptr3 = (int*)ptr2;
ptr3 + 1; // valid
ptr3[1];  // valid

char* ptr4 = (char*)ptr2;
ptr3 + 1; // valid
ptr3[1];  // valid
\end{lstlisting}}

  	\begin{frame}{Le pointeur \texttt{void*}}
  		\begin{block}{Le pointeur \texttt{void*}}
  			Le cast depuis et vers le pointeur \texttt{void*} à partir de tout autre type de pointeur se fait \alert{implicitement} dans le langage C.\\
  			Chaque type de pointeur peut être transformé depuis et vers le pointeur \texttt{void*}. Cependant, faire de l'arithmétique des pointeurs ou accéder aux éléments en utilisant le pointeur \texttt{void*} est \alert{interdit}. Il doit y avoir une conversion vers un type de pointeur différent que \texttt{void*} avant d'accéder ou d'effectuer de l'arithmétique.
  		\end{block}
  	\end{frame}
  
   	\begin{frame}{Le pointeur \texttt{void*}}
  		\begin{exampleblock}{Example : }
  			\voidPtrExmpl
  		\end{exampleblock}
  	\end{frame}
  	
\defverbatim[colored]\staticInFuncExmp{	
\begin{lstlisting}[language=C,tabsize=4]
// includes...
void toto()
{
	static int g = 5;
	int i = 5;
	g += 5;
	i += 5;
	printf("%d - %d\n", g, i);
}

int main() {
	for (int j = 0; j < 5; j++)
		toto();
}
\end{lstlisting}}

    \subsection{Le keyword static et extern}
    \begin{frame}{le mot-clé static}
    	\begin{block}{Définition :}
    		Le mot-clé static se comporte différemment selon la façon dont il est utilisé :
    		\begin{itemize}
    			\item Une variable déclaré statique à l'intérieur d'une fonction conserve sa valeur entre les appels. (La variable agit donc comme si elle était globale)
    			\item Une variable globale statique ou une fonction statique n'est "accessible" que dans le fichier dans lequel elle est déclarée. (un peu comme le mot-clé \alert{private} en Java)
    		\end{itemize}
    	\end{block}
    \end{frame}

	\begin{frame}{le mot-clé static}
		\framesubtitle{le mot-clé static : Example 1}
		\staticInFuncExmp
	\end{frame}

	\begin{frame}{le mot-clé static}
		\framesubtitle{le mot-clé static : Solution 1}
		\begin{itemize}
			\item 1iere itération : \texttt{10 - 10}
			\item 2ieme itération : \texttt{15 - 10}
			\item 3ieme itération : \texttt{20 - 10}
			\item 4ieme itération : \texttt{25 - 10}
			\item 5ieme itération : \texttt{30 - 10}
		\end{itemize}
	\end{frame}

\defverbatim[colored]\staticPrivExmplOne{	
\begin{lstlisting}[language=C,tabsize=4]
// file1.c
static int pix10 = 314;

static void foo()
{
	// ...
}
\end{lstlisting}}

\defverbatim[colored]\staticPrivExmplTwo{	
\begin{lstlisting}[language=C,tabsize=4]
// mainc.c		
int main()
{
	pix10 = 628;
	foo();
}
\end{lstlisting}}

	\begin{frame}{le mot-clé static}
		\framesubtitle{le mot-clé static : Example 2}
		\staticPrivExmplOne
		\staticPrivExmplTwo
	\end{frame}

\defverbatim[colored]\staticPrivExmplSolutionOne{	
\begin{lstlisting}[language=C,tabsize=4]
// file1.c
static int pix10 = 314; // this variable is only visible in this 
					    // translation unit (file1.c)
// this function can only be used in this translation unit
// foo is only visible in 'file1.c'
static void foo()						
{
	// ...
}
\end{lstlisting}}

\defverbatim[colored]\staticPrivExmplSolutionTwo{	
\begin{lstlisting}[language=C,tabsize=4]
// mainc.c		
int main()
{
	pix10 = 628; // undeclared variable pix10
	foo();		 // implicit declaration of function 'prv_func'
}
\end{lstlisting}}
	\begin{frame}{le mot-clé static}
		\framesubtitle{le mot-clé static : Solution 2}
		\staticPrivExmplSolutionOne
		\staticPrivExmplSolutionTwo
	\end{frame}
	\begin{frame}{le mot-clé extern}
		\begin{block}{Définition :}
			le mot-clé \alert{extern} est utilisé pour étendre la visibilité des variables / fonctions.
		\end{block}
		\begin{alertblock}{Quelques remarques : }
			- L'utilisation d'\alert{extern} avec des fonctions est redondant.\\
			- L'utilisation de \texttt{extern type var} amènera la variable \alert{\texttt{var}} déclaré dans un \alert{autre} fichier \alert{.c} dans le fichier \alert{.c} courant pendant l'édition du lien.
		\end{alertblock}
	\end{frame}
		
    
    \subsection{Les opérateurs et ordre d'évaluation}
    \begin{frame}{Priorité des opérateurs}
    	Table des opérateurs : \footnote[frame]{pour plus de détails: \url{https://en.cppreference.com/w/c/language/operator_precedence}}
  		\begin{table}[!b]
	  		{\carlitoTLF % Use monospaced lining figures
	  		\begin{tabularx}{\textwidth}{Xrrr}
	  			\textbf{Priorité} & \textbf{Les opérateurs} \\
	  			\toprule
	  			\texttt{1} & ++ -{}- () [] . ->    			\\
	  			\texttt{2} & ++ -{}- + - ! ~ * \& sizeof    \\
	  			\texttt{3} & * / \%  						\\
	  			\texttt{4} & + -							\\
	  			\texttt{5} & <{}< >{}>						\\
	  			\texttt{6} & <{}<= >{}>=					\\
	  			\texttt{7} & == !=							\\
	  			\texttt{8} & \&								\\
	  			\bottomrule
	  		\end{tabularx}}
	  		\caption{Priorité des opérateurs en C}
	  	\end{table}  	
  	\end{frame}
  
  	\begin{frame}{Priorité des opérateurs}
  		Table des opérateurs : \footnote[frame]{pour plus de détails: \url{https://en.cppreference.com/w/c/language/operator_precedence}}
  		\begin{table}[!b]
  			{\carlitoTLF % Use monospaced lining figures
  				\begin{tabularx}{\textwidth}{Xrrr}
  					\textbf{Priorité} & \textbf{Les opérateurs} \\
  					\toprule
  					\texttt{9} & \textasciicircum		\\
  					\texttt{10} & |						\\
  					\texttt{11} & \&\&					\\
  					\texttt{12} & ||					\\
  					\texttt{13} & ?:					\\
  					\texttt{14} & = += -= *= /= \%=	<{}<= >{}>= \&= \textasciicircum= |= \\
  					\texttt{15} & , \\
  					\bottomrule
  			\end{tabularx}}
  			\caption{Priorité des opérateurs en C}
  		\end{table}  	
  	\end{frame}
  
\defverbatim[colored]\opOrderUB{	
\begin{lstlisting}[language=C,tabsize=4]
i = (i++); 		// i = 0 intially
i = ++i + i++; 	// i = 0 intially
i = i++ + 1; 	// i = 0 intially
f(++i, ++i); 	// i = 0 intially
\end{lstlisting}}

\defverbatim[colored]\opOrderUBSolution{	
\begin{lstlisting}[language=C,tabsize=4]
i = (i++); 			// undefined behavior
i = ++i + i++; 		// undefined behavior
i = i++ + 1; 		// undefined behavior
f(++i, ++i); 		// undefined behavior
f(i = -1, i = -2); 	// undefined behavior
f(i, i++); 			// undefined behavior
a[i] = i++; 		// undefined bevahior
\end{lstlisting}}
  	\begin{frame}{Les opérateurs et ordre d'évaluation}
  		Pouvez-vous prédire la valeur de \texttt{i} sur chaque instruction sachant que \texttt{i} est initialement $0$ :
  		\opOrderUB
  	\end{frame}
  	
  	\begin{frame}{Les opérateurs et ordre d'évaluation}
  		\opOrderUBSolution
  		\begin{alertblock}{Explication :}
  			Pour une expression de type \texttt{ExpA \alert{op} ExpB} ou \alert{op} n'est pas \texttt{\alert{\&\&}} ou \texttt{\alert{||}} ou \texttt{\alert{,}} l'ordre d'evaluation de \texttt{ExpA} et \texttt{ExpB} est indéfini (c'est-à-dire que \texttt{ExpA} peut être évalué avant \texttt{ExpB} ou vice versa)
  		\end{alertblock}
  		\begin{exampleblock}{Example :}
  			\texttt{res = foo() + bar()} foo peut être appelé avant bar ou vice versa
  		\end{exampleblock}
  	\end{frame}
  	
\defverbatim[colored]\opNoUB{	
\begin{lstlisting}[language=C,tabsize=4]
int res = f1() && f2(); // f1() is called before f2()
int res = f1() || f2;   // f1() is called before f2()
int res = f1(), f2();   // f1() is called before f2()
int res = cond() ? f1() : f2() // cond() is called then f1() 
							   // is called if the result of 
							   // cond() is true otherwise f2() 
							   // is called
\end{lstlisting}}

    \begin{frame}{Les opérateurs et ordre d'évaluation}
	  	\begin{alertblock}{Opérateurs qui garantissent l'ordre}
	  		Les opérateurs \texttt{\alert{\&\&}}, \texttt{\alert{||}} et \texttt{\alert{,}} garantissent que l'expression à gauche est évaluée avant l'expression à droite.\\
	  		L'opérateur ternaire (\texttt{\alert{?:}}) garantit que la condition est évaluée en premier. Si la condition est vraie, l'expression de gauche à \alert{:} est évaluée, sinon l'expression à droite de \alert{:} est évaluée.
	  	\end{alertblock}
  		\begin{exampleblock}{Example :}
  			\opNoUB
  		\end{exampleblock}
  	\end{frame}

\defverbatim[colored]\opNoUB{	
\begin{lstlisting}[language=C,tabsize=4]
a[i] = i;   		// OK
a[i] = j;   		// OK
a[i++] = i; 		// undefined behavior
a[i] = i++; 		// undefined behavior
int p = i++ && i--; // OK
int p = i++ || i--; // OK
int p = i++, i--; 	// OK
\end{lstlisting}}
	\begin{frame}{Les opérateurs et ordre d'évaluation}
		\begin{alertblock}{Pour éviter complètement le problème :}
			Une règle de base qui devrait nous protéger de ce genre de comportement indéfini est de ne pas modifier et utiliser la même variable dans la même instruction, c'est-à-dire éviter de changer et de lire à partir de la même variable avant un \alert{point virgule ';'}.
		\end{alertblock}
		\begin{exampleblock}{Example :}
			\opNoUB
		\end{exampleblock}
	\end{frame}