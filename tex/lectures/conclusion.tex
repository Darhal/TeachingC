\section{Conclusion}
\begin{frame}{Conclusion}
	\framesubtitle{Les dix commandements pour réussir en C}
	\begin{enumerate}
		% \item Undefined behaviour, avoid them.
		\item \alert{Les comportements indéfinis}, évitez-les.
		% \item Allocated memory bounds, don't exceed them.
		\item \alert{Les limites de la mémoire allouées}, ne les dépassez pas.
		% \item Your variables, always initialize them.
		\item \alert{Vos variables}, initialisez-les toujours.
		% \item Memory leaks, eliminate them.	
		\item \alert{Les fuites de mémoire}, éliminez-les.
		% \item Arrays and pointers, don't confuse between them.
		\item \alert{Les tableaux et les pointeurs}, ne les confondez pas.
		% \item Unions, becareful while manipulating them.
		\item \alert{Les unions}, faites attention lorsque vous les utilisez.
		% \item Your data, be aware where they live.
		\item \alert{Vos variables}, sachez dans quelle région elles résident.
		% \item Pointer casting, be careful when doing them.
		\item \alert{Pointer casting}, soyez prudent lorsque vous les éffectuer.
		% \item Your strings, don't forget to null terminate them
		\item \alert{Vos chaînes}, n'oubliez pas de les terminer par null
		% \item The tools, don't hesitate to use them.
		\item \alert{Les outils}, n'hésitez pas à les utiliser.
	\end{enumerate}
\end{frame}